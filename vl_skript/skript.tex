\documentclass[numbers=endperiod]{scrartcl}
\usepackage{tikz}
% Set \germantrue if the document is in German language
\newif\ifgerman
%\germanfalse
\germantrue
\ifgerman
  \def\babellanguage{ngerman}
  % Most fonts do not advertise a specialised
  % Latin/English (latn.ENG) or
  % Latin/German 
  % but only Latin (latn) script.
  % Thus, when loading fonts, ask for the generic version.
  \def\babelfontlanguage{Default}
  \def\babelscript{Latin}

  \def\biblatexsorting{de_DE}
  \def\namecorollary{Korollar}
  \def\namedefinition{Definition}
  \def\nameexercise{Aufgabe}
  \def\namelemma{Lemma}
  \def\nameproposition{Aussage}
  \def\nameremark{Anmerkung}
  \def\nametheorem{Satz}
\else
  \def\babellanguage{english}
  \def\babelfontlanguage{Default}
  \def\babelscript{Latin}
  \def\biblatexsorting{en_US}
  \def\namecorollary{Corollary}
  \def\namedefinition{Definition}
  \def\nameexercise{Exercise}
  \def\namelemma{Lemma}
  \def\nameproposition{Proposition}
  \def\nameremark{Remark}
  \def\nametheorem{Theorem}
\fi

\usepackage{amsmath}
\usepackage{amsfonts}
\usepackage{amssymb}
\usepackage{amsthm}
% You can declare additional operators, eg.
%\DeclareMathOperator{\rem}{rem}

\usepackage{iftex}
\ifPDFTeX
  % Warning: Inputenc is only a hack to teach pdfTeX Unicode.  Beware of dragons!
  \usepackage[utf8]{inputenc}
  \usepackage[T1]{fontenc}
  \usepackage{lmodern}
  \usepackage[\babellanguage]{babel}
\else
  \usepackage{fontspec}
  \usepackage{babel}

  \babelprovide[import,
    language=\babelfontlanguage,
    script=\babelscript,
    main]{\babellanguage}
  \usepackage{unicode-math}
  % Load the Computer Modern Unicode OTF fonts from the system. Also see README.md 
  \babelfont{rm}%
    % Use non-lining numerals for non-math text
    [Numbers=OldStyle]
    {CMU Serif}
  \babelfont{sf}{CMU Sans Serif}
  \babelfont{tt}{CMU Typewriter Text}

  % Make it look more like CM Math
  % % https://tex.stackexchange.com/q/140754/90407
  \setmathfont{Latin Modern Math}
  \setmathfont[range=\mathbb]{TeX Gyre Termes Math}

  % LM Math only provides the U+2216 code point for "set minus", however,
  % unicode-math maps \setminus the "reverse solidus operator" U+29F5 which LMM
  % doesn't provide, instead mapping \smallsetminus to U+2216.  However, whether
  % you want a "small" set minux or a "long" is a font decision (IMHO), so
  % remap \setminus to the correct Unicode code point.
  % https://tex.stackexchange.com/a/140343/90407
  % https://tex.stackexchange.com/a/161873/90407
  \ExplSyntaxOn
  \AtBeginDocument
  {
     \__um_process_symbol_noparse:nnn {"02216}{\setminus}{\mathbin}
  }
  \ExplSyntaxOff
\fi

% Protruding punctuation
\usepackage{microtype}

% \enquote{} for setting things in "quotes"
\usepackage{csquotes}

% Additional [H] specifier for floats that actually shouldn't be floats
% but still be considered figures
\usepackage{float}

% To create custom itemizations, eg.
% \begin{itemize}[label=\alph*)]
\usepackage{enumitem}

% Typesetting code, setup for C by default
\usepackage{xcolor}
\usepackage{listings}
\lstdefinestyle{default}{%
  numbers=left,
  stepnumber=1,
  numberstyle=\tiny,
  basicstyle=\ttfamily,
  backgroundcolor=\color{gray!8},
  commentstyle=\color{green!60!blue}\itshape,
  keywordstyle=\color{blue},
  stringstyle=\color{blue!30!red},
  tabsize=4,
  keepspaces=true,
}
\lstset{style=default, language=C}

% Break URLs at hyphens
\PassOptionsToPackage{hyphens}{url}
% Make PDF links work with non-latin languages via Unicode
\usepackage[pdfusetitle,pdfencoding=auto]{hyperref}
\pdfstringdefDisableCommands{
  \def\@{}
}

% Typeset units with
% \SI{8}{\bit} or \si{\bit}
\usepackage[
  binary-units=true,
  % Make \num{42} and $\num{42}$ behave differently.
  detect-mode,
  detect-family,
  detect-inline-family=math,
]{siunitx}

% Disable warning for biblatex < 3.14
\makeatletter\def\blx@nowarnpolyglossia{}\makeatother
\usepackage[
  natbib=true,
  backend=biber,
  style=numeric,
  citestyle=numeric,
  sorting=nty,
  sortlocale=\biblatexsorting,
  seconds=true,
  alldates=iso,
]{biblatex}
\addbibresource{./sources.bib}

% Headers & Footers
\usepackage[headsepline]{scrlayer-scrpage}
\pagestyle{scrheadings}
% titling conflicts with \maketitle etc. from KOMA, but as long as we don't use
% that this is fine.
\usepackage{titling}
\lohead*{\theauthor}
\cohead*{\thetitle}
%\rohead*{\today}
\cofoot*{\thepage}

% Blind text
\usepackage{lipsum}



\begin{document}

\author{}
\title{Vorlesungsskript -- Technische und logische Grundlagen der Informatik}
% Add some neat environments for scripts and exercises

\theoremstyle{definition}
\newtheorem{definition}{\namedefinition}

\theoremstyle{remark}
\newtheorem{remark}{\nameremark}

\theoremstyle{plain}
\newtheorem{proposition}{\nameproposition}
\newtheorem{theorem}{\nametheorem}
\newtheorem{lemma}{\namelemma}
\newtheorem{corollary}{\namecorollary}

\newtheoremstyle{exercise}% name of the style to be used
  {}         % measure of space to leave above the theorem. E.g.: 3pt
  {}         % measure of space to leave below the theorem. E.g.: 3pt
  {\upshape} % name of font to use in the body of the theorem
  {}         % measure of space to indent
  {\bfseries}% name of head font
  {}         % punctuation between head and body
  { }        % space after theorem head; " " = normal interword space
  {%
\thmname{#1}\thmnumber{ #2}.\thmnote{ (#3)}%  Manually specify head
}
\theoremstyle{exercise}
\newtheorem{exercise}{\nameexercise}[section]


\begin{center}
\Huge{\textbf{Vorlesungsskript}}\\
\Large{\textbf{Technische und logische Grundlagen der Informatik}}\\
\large{Version: Alpha v.0.1}
\end{center}

\tableofcontents

% Example content
\section{Signale im Zeit und Frequenzbereich}
Folgender Abschnitt soll vor allem als Einleitung in das Thema dienen, da hier grundlegend erklärt wird, wie die Bits und Bytes auf dem Computer überhaupt entstehen bzw. übertragen werden können. Eine gute Quelle zur weiterführenden Literatur bietet: \cite{kemnitz2009technische1}

\subsection{Signal}
\begin{itemize}
	\item Signal is the physical representation of data
	\item Analog signal is a sequence of continuous values
	\item Digital signal is a sequence of discrete values
	\item Data is converted to signal which is sent over a transmission channel
	\item Transmission channel = access points + physical medium carrying signal
	\item The need to convert - Quantization
	\begin{itemize}
		\item Computers can only deal with digital data => discrete signal
		\item Physical mediums are by nature analog => continuous signal
		\item Must convert from digital signal to analog signal (and vice versa)
	\end{itemize}
	\item The need to measure -- Sampling
	\begin{itemize}
		\item Computers can only deal with discrete time
		\item Physical mediums’ state vary continuously
		\item Must rely on periodical measurements of the physical medium
	\end{itemize}
\end{itemize}

 \begin{tikzpicture}
    \draw (0,0) -- (12,0);
    \draw (0.2,1)node[left,font=\tiny] {$y=1$} -- (11.8,1);
    \draw (0.2,-1)node[left,font=\tiny] {$y=-1$} -- (11.8,-1); 
    \foreach \x in {0,0.5,...,12}{
    \draw (\x,-0.2)node [below,font=\tiny,] {\x} -- (\x,0.2) ;
    }
    \draw[ultra thick, red] (3,0) sin (4,1);    %% the real business in this line
    \draw[ultra thick, blue] (4,1) cos (5,0);    %% the real business in this line
    \draw[ultra thick, red] (5,0) sin (6,-1);    %% the real business in this line
    \draw[ultra thick, blue] (6,-1) cos (7,0);    %% the real business in this line
    \draw[ultra thick, red] (7,0)  sin (8,1);    %% the real business in this line
    \draw[ultra thick, blue] (8,1) cos (9,0);    %% the real business in this line
    \draw[ultra thick, red] (9,0) sin (10,-1);    %% the real business in this line
    \draw[ultra thick, blue] (10,-1) cos (11,0);    %% the real business in this line
    \end{tikzpicture}

\subsection{Grundlegende Signalverarbeitung}
\subsection{Periodische Signale}

\subsubsection{Signalkomposition}

\section{Boolesche Algebra}

\section{Zahlenkodierung \& Zahlendarstellung}
\begin{itemize}
	\item Natürliche Zahlen ($\mathbb{N}$)
	\begin{itemize}
		\item Konstruktion durch Vervollständigung
		\item Ziffernsysteme
	\end{itemize}
	\item Positionssysteme \& Positionssystem zur Basis $q$
	\item Zifferndarstellung
	\item Dualdarstellung \& Realisierung
	\item Ganze Zahlen ($\mathbb{Z}$)
	\begin{itemize}
		\item Konstruktion durch Abschluß von $\mathbb{N}$ unter Subtraktion
		\item Zifferndarstellung \& Realisierung
		\item Ganze Zahlen variabler Länge
	\end{itemize}
	\item Rationale Zahlen ($\mathbb{Q}$)
	\begin{itemize}
		\item Konstruktion durch Abschluß von $\mathbb{Z}$ unter Division
		\item Zifferndarstellung 
		\item Dezimal- und Dualbrüche
		\item Praktische Realisierung
	\end{itemize}
	\item Reelle Zahlen ($\mathbb{R}$)
	\begin{itemize}
		\item Konstruktion durch Vervollständigung von $\mathbb{Q}$
		\item Abzählbarkeit und Zifferndarstellung
		\item Absoluter und relativer Fehler
		\item Konstruktion Gleitkommazahlen
		\item Gleitkommazahlen und Rundungsfehler
		\item Praktische Realisierung
		\item Gleitkommaarithmetik
	\end{itemize}
\end{itemize}

\begin{theorem}[Gauß Summe]

\end{theorem}
\begin{proof}
Der Beweis obliegt dem Leser als Übungsaufgabe.
\end{proof}

\begin{exercise}[Programmargumente ausgeben]
Das folgende Programm iteriert über die Konsolenargumente, inklusive des
Programmaufrufs:
\begin{lstlisting}[style=default]
int main(int argc, char *argv[])
{
	/* Loops over program arguments */
	for (int i = 0; i <= argc; i++) {
		printf("Argument %d: \"%s\"\n", argv[i]);
	}
}
\end{lstlisting}
\end{exercise}

% End Example content

\begin{appendix}
\printbibliography
\end{appendix}

\end{document}
