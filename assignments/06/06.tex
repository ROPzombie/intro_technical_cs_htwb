%start preamble
\documentclass[paper=a4,fontsize=11pt]{scrartcl}%kind of doc, font size, paper size

\usepackage{fontspec}
\defaultfontfeatures{Ligatures=TeX}
%\setsansfont{Liberation Sans}
\usepackage{polyglossia}	
\setdefaultlanguage[spelling=new, babelshorthands=true]{german}
\usepackage{csquotes}
		
\usepackage{amsmath}%get math done
\usepackage{amsthm}%get theorems and proofs done
\usepackage{graphicx}%get pictures & graphics done
\graphicspath{{pictures/}}%folder to stash all kind of pictures etc
\usepackage{hyperref}%for links to web
\usepackage{amssymb}%symbolics for math
\usepackage{amsfonts}%extra fonts
\usepackage []{natbib}%citation style
\usepackage{caption}%captions under everything
\usepackage{listings}
\usepackage[titletoc]{appendix}
\numberwithin{equation}{section} 
\usepackage[printonlyused,withpage]{acronym}%how to handle acronyms
\usepackage{float}%for garphics and how to let them floating around in the doc
\usepackage{cclicenses}%license!
\usepackage{xcolor}%nicer colors, here used for links
\usepackage{wrapfig}%making graphics floated by text and not done by minipage
\usepackage{dsfont}
\usepackage{fancyhdr}
\usepackage{enumitem}



%settings colors for links
\hypersetup{
    colorlinks,
    linkcolor={blue!50!black},
    citecolor={blue},
    urlcolor={blue!80!black}
}

\definecolor{pblue}{rgb}{0.13,0.13,1}
\definecolor{pgreen}{rgb}{0,0.5,0}
\definecolor{pred}{rgb}{0.9,0,0}
\definecolor{pgrey}{rgb}{0.46,0.45,0.48}

\pagestyle{fancy}
\lhead{LtGdI -- Übung\\ Wintersemester 2021/22}
\rhead{FB-4 -- AI\\ HTW-Berlin}
%\lfoot{01}
%\cfoot{}
%\fancyfoot[R]{\thepage}
\renewcommand{\headrulewidth}{0.4pt}
%\renewcommand{\footrulewidth}{0.4pt}

\lstdefinestyle{Bash}{
  language=bash,
  showstringspaces=false,
  basicstyle=\small\sffamily,
  numbers=left,
  numberstyle=\tiny,
  numbersep=5pt,
  frame=trlb,
  columns=fullflexible,
  backgroundcolor=\color{gray!20},
  linewidth=0.9\linewidth,
  %xleftmargin=0.5\linewidth
  upquote=true,
  columns=fullflexible,
  literate={*}{{\char42}}1
         {-}{{\char45}}1
}

\lstdefinelanguage
   [x86]{Assembler}
   [x86masm]{Assembler} % based on the "x86masm" dialect
   % with these extra keywords:
   {morekeywords={CDQE,CQO,CMPSQ,CMPXCHG16B,JRCXZ,LODSQ,MOVSXD, %
                  POPFQ,PUSHFQ,SCASQ,STOSQ,IRETQ,RDTSCP,SWAPGS, %
                  eax,edx,ecx,ebx,esi,edi,esp,ebp, %
                  e8,e8d,e8w,e8b,e9,e9d,e9w,e9b, %
                  e10,e10d,e10w,e10b,e11,e11d,e11w,e11b, %
                  e12,e12d,e12w,e12b,e13,e13d,e13w,e13b, %
                  e14,e14d,e14w,e14b,e15,e15d,e15w,e15b}} %

\lstset{language=[x86]Assembler}


\lstset{language=C,
	basicstyle=\ttfamily,
   	keywordstyle=\color{blue}\ttfamily,
    stringstyle=\color{red}\ttfamily,
    commentstyle=\color{cyan}\ttfamily,
    morecomment=[l][\color{magenta}]{\#}
    showstringspaces=false,
  	basicstyle=\small\sffamily,
  	numbers=left,
  	numberstyle=\tiny,
  	numbersep=5pt,
  	frame=trlb,
  	upquote=true,
  	columns=fullflexible,
  	backgroundcolor=\color{gray!20},
  	%linewidth=0.9\linewidth,
  	literate=*{*}{\normalfont{*}}1,
}


\newcommand{\specialcell}[2][c]{%
  \begin{tabular}[#1]{@{}c@{}}#2\end{tabular}}

%%here begins the actual document%%
\newcommand{\horrule}[1]{\rule{\linewidth}{#1}} % Create horizontal rule command with 1 argument of height


\DeclareMathOperator{\id}{id}

\title{	
\normalfont \normalsize 
\textsc{Übungsblatt 01}
}
\begin{document}
\vspace*{-1cm}
\begin{center}
\Large{\textbf{Übungsblatt 5}}
\end{center}

\begin{center}\Large{\textbf{Aufgabe A -- Gleitkommazahlen}}\end{center}
Beantworte Sie folgende Fragen:
\begin{enumerate}
	\item Warum werden i.d.R. Fließkommazahlen und nicht Festkommazahlen zur Darstellung von rationalen Zahlen verwendet? Anders formuliert: Welche Vorteile bringt diese Art der Darstellung?
	\item Für eine Fließkommazahl-Darstellung werden die gegebenen Bits in drei Abschnitte unterteilt. Wie heißen diese Abschnitte?
	\item Welchen Vorteil bringt es wenn dem Exponenten mehr Bits zugeteilt werden bzw. welchen Vorteil bringt es wenn der Mantisse mehr Bits zugeteilt werden?
	\item Erklären sie Unterlauf (Underflow) und Überlauf (Overflow) für Fließkommazahlen.
	\item Nennen sie Beispiele für Festlegungen, die der IEEE 754 Standard mitbringt. Warum ist eine solche Standardisierung sinnvoll?
	\item Was ist die betragsmäßig größte bzw. kleinste darstellbare Zahl im IEEE-754 32bit Standard? Geben sie auch die Bits an.
	\item Berechnen sie die Gleitkommazahlen:
	\begin{enumerate}
		\item 18,6125
		\item 23,42
	\end{enumerate}
\end{enumerate}

\begin{center}\Large{\textbf{Aufgabe B -- Gleitkommazahlen in C}}\end{center}
Gegeben sei folgender C Code (in Moodle auch als Sourcecode).
\begin{lstlisting}[language=C]
#include  <stdio.h>
#include  <ctype.h>
union zahl {
        float   float_zahl;
        double  double_zahl;
        struct {
           unsigned bitmuster1;
           unsigned bitmuster2;
        } long_format;
};
union zahl  z1, z2;

void  dual_ausgabe(unsigned long zahl, int strich1, int strich2) {
   int i;
   for (i=sizeof(long)*8-1; i>=0; i--)
      printf("%ld%s", (zahl>>i) & 1, (i==strich1 || i==strich2) ? " | " : "");
}

int  main(void)
{
   printf("Gib Deine Gleitpunktzahl ein: ");
   scanf("%lf", &z1.double_zahl);

   z2.float_zahl = z1.double_zahl;
   printf("double: ");
   dual_ausgabe(z1.long_format.bitmuster2, 31, 20);
   dual_ausgabe(z1.long_format.bitmuster1, -1, -1);
   printf("\n float: ");
   dual_ausgabe(z2.long_format.bitmuster1, 31, 23);
   printf("\n");

   return 0;
}
\end{lstlisting}

\begin{enumerate}
	\item Kompilieren sie den obigen Code und Testen sie ihn mit den aus Aufgabenteil A umgerechneten Zahlen.
	\item Lesen sie den code und versuchen sie den Code zunächst zu verstehen.
	\item Erläutern sie die Umrechnung von Dezimal in die beiden Gleitkommaformate (float, double) nach IEEE 754.
\end{enumerate}

\end{document}