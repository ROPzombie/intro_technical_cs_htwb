%start preamble
\documentclass[paper=a4,fontsize=11pt]{scrartcl}%kind of doc, font size, paper size

\usepackage{fontspec}
\defaultfontfeatures{Ligatures=TeX}
%\setsansfont{Liberation Sans}
\usepackage{polyglossia}	
\setdefaultlanguage[spelling=new, babelshorthands=true]{german}
\usepackage{csquotes}
		
\usepackage{amsmath}%get math done
\usepackage{amsthm}%get theorems and proofs done
\usepackage{graphicx}%get pictures & graphics done
\graphicspath{{pictures/}}%folder to stash all kind of pictures etc
\usepackage{hyperref}%for links to web
\usepackage{amssymb}%symbolics for math
\usepackage{amsfonts}%extra fonts
\usepackage []{natbib}%citation style
\usepackage{caption}%captions under everything
\usepackage{listings}
\usepackage[titletoc]{appendix}
\numberwithin{equation}{section} 
\usepackage[printonlyused,withpage]{acronym}%how to handle acronyms
\usepackage{float}%for garphics and how to let them floating around in the doc
\usepackage{cclicenses}%license!
\usepackage{xcolor}%nicer colors, here used for links
\usepackage{wrapfig}%making graphics floated by text and not done by minipage
\usepackage{dsfont}
\usepackage{stmaryrd}
\usepackage{geometry}
\usepackage{fancyhdr}
\usepackage{menukeys}
\usepackage{enumitem}



%settings colors for links
\hypersetup{
    colorlinks,
    linkcolor={blue!50!black},
    citecolor={blue},
    urlcolor={blue!80!black}
}

\definecolor{pblue}{rgb}{0.13,0.13,1}
\definecolor{pgreen}{rgb}{0,0.5,0}
\definecolor{pred}{rgb}{0.9,0,0}
\definecolor{pgrey}{rgb}{0.46,0.45,0.48}

\pagestyle{fancy}
\lhead{LGtI -- Übung\\ Wintersemester 2021/22}
\rhead{FB-4 -- AI\\ HTW-Berlin}
%\lfoot{01}
%\cfoot{}
%\fancyfoot[R]{\thepage}
\renewcommand{\headrulewidth}{0.4pt}
%\renewcommand{\footrulewidth}{0.4pt}

\lstdefinestyle{Bash}{
  language=bash,
  showstringspaces=false,
  basicstyle=\small\sffamily,
  numbers=left,
  numberstyle=\tiny,
  numbersep=5pt,
  frame=trlb,
  columns=fullflexible,
  backgroundcolor=\color{gray!20},
  linewidth=0.9\linewidth,
  %xleftmargin=0.5\linewidth
  upquote=true,
  columns=fullflexible,
  literate={*}{{\char42}}1
         {-}{{\char45}}1
}

\lstdefinelanguage
   [x86]{Assembler}
   [x86masm]{Assembler} % based on the "x86masm" dialect
   % with these extra keywords:
   {morekeywords={CDQE,CQO,CMPSQ,CMPXCHG16B,JRCXZ,LODSQ,MOVSXD, %
                  POPFQ,PUSHFQ,SCASQ,STOSQ,IRETQ,RDTSCP,SWAPGS, %
                  eax,edx,ecx,ebx,esi,edi,esp,ebp, %
                  e8,e8d,e8w,e8b,e9,e9d,e9w,e9b, %
                  e10,e10d,e10w,e10b,e11,e11d,e11w,e11b, %
                  e12,e12d,e12w,e12b,e13,e13d,e13w,e13b, %
                  e14,e14d,e14w,e14b,e15,e15d,e15w,e15b}} %

\lstset{language=[x86]Assembler}


\lstset{language=C,
	basicstyle=\ttfamily,
   	keywordstyle=\color{blue}\ttfamily,
    stringstyle=\color{red}\ttfamily,
    commentstyle=\color{cyan}\ttfamily,
    morecomment=[l][\color{magenta}]{\#}
    showstringspaces=false,
  	basicstyle=\small\sffamily,
  	numbers=left,
  	numberstyle=\tiny,
  	numbersep=5pt,
  	frame=trlb,
  	upquote=true,
  	columns=fullflexible,
  	backgroundcolor=\color{gray!20},
  	%linewidth=0.9\linewidth,
  	literate=*{*}{\normalfont{*}}1,
}


\newcommand{\specialcell}[2][c]{%
  \begin{tabular}[#1]{@{}c@{}}#2\end{tabular}}

%%here begins the actual document%%
\newcommand{\horrule}[1]{\rule{\linewidth}{#1}} % Create horizontal rule command with 1 argument of height


\DeclareMathOperator{\id}{id}

\title{	
\normalfont \normalsize 
\textsc{Übungsblatt 01}
}
\begin{document}
\vspace*{-1cm}
\begin{center}
\Large{\textbf{Übungsblatt 2}}
\end{center}

\begin{center}\Large{\textbf{Aufgabe A -- KNF \& DNF}}\end{center}

\begin{enumerate}
	\item Finden sie die disjunktive Normalform und die konjunktive Normalform zur folgenden aussagenlogischen Formel:
	$$ 
		(A \lor B) \Rightarrow C)
	$$ 
	\item Finden sie die disjunktive Normalform und die konjunktive Normalform zur folgenden aussagenlogischen Formel: 
	$$
	(A \land B) \Leftrightarrow C)
	$$
	\item Finden Sie die disjunktive Normalform und die konjunktive Normalform zur folgenden aussagenlogischen Formel: 
	$$
	(A \lor B) \Rightarrow \neg C) \Rightarrow D
	$$
\end{enumerate}

\begin{center}\Large{\textbf{Aufgabe B -- Bit Hacks in C}}\end{center}
	Folgenden Code soll kompiliert und ausgeführt werden.
	\begin{lstlisting}[style=Bash, language=Bash]
gcc program_name -o compiled_name
\end{lstlisting}
	\begin{enumerate}
		\item Kompilieren sie das Programm \emph{p1.c} und führen sie dies für natürliche Zahlen aus.
		\item Testen sie verschiedene Eingaben. Stellen sie eine erste Vermutung auf!
		\item Schauen sie in den Quellcode -- nur die \texttt{main}-Funktion: Der Ausdruck \texttt{f = (v \& (v - 1)) == 0;} ist relevant. Der Rest dient der Ein- und Ausgabe.
		\item Nehmen sie eine Wert und gehen den logischen Ausdruck Schritt für Schritt durch.
		\lstinputlisting[language=C]{src/check_power2.c}
		\item Kompilieren sie das Programm \emph{p2.c} und führen sie dies für natürliche Zahlen aus.
		\item Testen sie verschiedene Eingaben. Stellen sie eine erste Vermutung auf!
		\item Schauen sie in den Quellcode -- nur die \texttt{main}-Funktion: Der Ausdruck \texttt{r = (n  (n \string^  - 1)) * v;} ist relevant. Der Rest dient der Ein- und Ausgabe.
		\lstinputlisting[language=C]{src/negate.c}
		\item Nehmen sie eine Wert und gehen den logischen Ausdruck Schritt für Schritt durch.
	\end{enumerate}
	
\end{document}