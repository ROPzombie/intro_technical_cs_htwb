%start preamble
\documentclass[paper=a4,fontsize=11pt]{scrartcl}%kind of doc, font size, paper size

\usepackage{fontspec}
\defaultfontfeatures{Ligatures=TeX}
%\setsansfont{Liberation Sans}
\usepackage{polyglossia}	
\setdefaultlanguage[spelling=new, babelshorthands=true]{german}
\usepackage{csquotes}
		
\usepackage{amsmath}%get math done
\usepackage{amsthm}%get theorems and proofs done
\usepackage{graphicx}%get pictures & graphics done
\graphicspath{{pictures/}}%folder to stash all kind of pictures etc
\usepackage{hyperref}%for links to web
\usepackage{amssymb}%symbolics for math
\usepackage{amsfonts}%extra fonts
\usepackage []{natbib}%citation style
\usepackage{caption}%captions under everything
\usepackage{listings}
\usepackage[titletoc]{appendix}
\numberwithin{equation}{section} 
\usepackage[printonlyused,withpage]{acronym}%how to handle acronyms
\usepackage{float}%for garphics and how to let them floating around in the doc
\usepackage{cclicenses}%license!
\usepackage{xcolor}%nicer colors, here used for links
\usepackage{wrapfig}%making graphics floated by text and not done by minipage
\usepackage{dsfont}
\usepackage{stmaryrd}
\usepackage{geometry}
\usepackage{fancyhdr}
\usepackage{menukeys}
\usepackage{enumitem}



%settings colors for links
\hypersetup{
    colorlinks,
    linkcolor={blue!50!black},
    citecolor={blue},
    urlcolor={blue!80!black}
}

\definecolor{pblue}{rgb}{0.13,0.13,1}
\definecolor{pgreen}{rgb}{0,0.5,0}
\definecolor{pred}{rgb}{0.9,0,0}
\definecolor{pgrey}{rgb}{0.46,0.45,0.48}

\pagestyle{fancy}
\lhead{LGtI -- Übung\\ Wintersemester 2021/22}
\rhead{FB-4 -- AI\\ HTW-Berlin}
%\lfoot{01}
%\cfoot{}
%\fancyfoot[R]{\thepage}
\renewcommand{\headrulewidth}{0.4pt}
%\renewcommand{\footrulewidth}{0.4pt}

\lstdefinestyle{Bash}{
  language=bash,
  showstringspaces=false,
  basicstyle=\small\sffamily,
  numbers=left,
  numberstyle=\tiny,
  numbersep=5pt,
  frame=trlb,
  columns=fullflexible,
  backgroundcolor=\color{gray!20},
  linewidth=0.9\linewidth,
  %xleftmargin=0.5\linewidth
  upquote=true,
  columns=fullflexible,
  literate={*}{{\char42}}1
         {-}{{\char45}}1
}


%%here begins the actual document%%
\newcommand{\horrule}[1]{\rule{\linewidth}{#1}} % Create horizontal rule command with 1 argument of height


\DeclareMathOperator{\id}{id}

\title{	
\normalfont \normalsize 
\textsc{Übungsblatt 01}
}
\begin{document}
\vspace*{-1cm}
\begin{center}
\Large{\textbf{Übungsblatt 1}}
\end{center}
\vspace*{-1cm}
\begin{center}\Large{\textbf{Aufgabe A -- Bool'sche Algebra}}\end{center}
\vspace*{-.8cm}
\begin{enumerate}
	\item In der Schaltalgebra gelten die folgenden alternativen Absorptionsgesetze:
	\begin{align*}
		(x \lor \neg y) \land y &= x \land y\\
		(x \land \neg y) \lor y &= x \lor y
	\end{align*}
	\begin{enumerate}
		\item Beweisen sie die Behauptung, indem sie die nachstehenden Wahrheitstabellen ergänzen:
	\begin{center}
	\begin{table}[ht]
	\centering
\begin{tabular}{|c|c|c|c|c|c|c|c|c|c|}
\hline
 $x$ & $y$ & $\neg x$ & $\neg y$ & $x \lor \neg y$ & $x \land \neg y$ & $(x \lor \neg y) \land y$ & $x \land y$ & $(x \land \neg y) \lor y$ & $x \lor y$ \\ \hline
0 & 0 &  &  &  &  &  &  &  &  \\ \hline
0 & 1 &  &  &  &  &  &  &  &  \\ \hline
1 & 0 &  &  &  &  &  &  &  &  \\ \hline
1 & 1 &  &  &  &  &  &  &  &  \\ \hline
\end{tabular}
\end{table}
	\end{center}
		\item Führen sie den Beweis erneut, diesmal aber auf algebraische Weise (d.h. durch Umformungen).
		\item Übertragen sie die Gesetze in die Sprache der Mengenalgebra. Sind sie dort auch gültig?
		\end{enumerate}
	\item Zeigen Sie, dass die booleschen Ausdrücke
	\begin{align*}
		\Phi &= x \land y  \lor \neg ((x \lor \neg y) \land y)\\
		\Psi &= \neg (x \land y) \lor x \lor y
	\end{align*}
	Tautologien sind, indem sie
	\begin{enumerate}
		\item für beide Funktionen eine Wahrheitstafel aufstellen,
		\item den Beweis durch algebraische Umformung führen.
	\end{enumerate}
	\item Bildet das Tripel $( \mathcal{V}, \cdot, +)$ mit
	\begin{align*}
		\mathcal{V} & := \{1,2,3,6\}\\
		\cdot 		& := kgV \text{ (kleinstes gemeinsames Vielfaches)}\\
		+ 			& := ggT \text{(größter gemeinsamer Teiler)}
	\end{align*}	
	eine boolesche Algebra?
	\item Vereinfachen sie die folgenden bool'schen Ausdrücke so weit wie möglich durch die Anwendung der algebraischen Umformungsregeln.
	\begin{enumerate}
		\item $ x_1 \overline{x_2 x_3 x_4} \lor x_1 x_2 \overline{x_3 x_4} \lor x_1 x_2 \overline{x_3} x_4 \lor x_1 \overline{x_2 x_3} x_4 \lor \overline{x_1} x_2 \overline{x_3 x_4} \lor x_1 x_2 x_3 \overline{x_4} \lor \overline{x_1} x_2 x_3 \overline{x_4}$
	\end{enumerate}
	\item Zeigen oder widerlegen sie die folgende Beziehung zwischen den Operatoren $\Leftrightarrow$ und $\not\Leftrightarrow$:
	\begin{enumerate}
		\item $x \Leftrightarrow y \Leftrightarrow z = x \not\Leftrightarrow y \not\Leftrightarrow z$
	\end{enumerate}
	\item Zeigen sie, dass die folgenden Varianten des Distributivgesetzes für $\Leftrightarrow$ und $\neg \Leftrightarrow$ falsch sind:
	\begin{enumerate}[resume]
		\item $(x \lor z) \neg \Leftrightarrow (y \lor z) = (x \neg \Leftrightarrow y) \lor z$
		\item $(x \land z) \Leftrightarrow (y \land z) = (x \Leftrightarrow y) \land z$
	\end{enumerate}
	\item Nachstehend sind die erweiterten De Morgan'schen Regeln aufgeführt.
	\begin{enumerate}
		\item $\overline{x_1 \land x_2 \land \ldots \land x_n} = \overline{x_1} \lor \overline{x_2} \lor \ldots \lor \overline{x_n}$
		\item $\overline{x_1 \lor x_2 \lor \ldots \lor x_n} = \overline{x_1} \land \overline{x_2} \land \ldots \land \overline{x_n}$
	\end{enumerate}
	\item Gegeben seien die folgenden drei bool'schen Funktionen:
	\begin{enumerate}
		\item $\varphi_1 := (x \Rightarrow y) \Rightarrow z$
		\item $\varphi_2 := x \Rightarrow (y \Rightarrow z)$
		\item $\varphi_3 := \overline{x \land y} \lor \overline{x \land \overline{z}}$
	\end{enumerate}
	Stellen sie $\varphi_1$ unter ausschließlicher Verwendung der NOR-Funktion, $\varphi_2$ unter ausschließlicher Verwendung der NAND-Funktion und $\varphi_3$ unter ausschließlicher Verwendung der Implikation dar.
	\item Zeigen sie unter Anwendung der Regeln der bool'schen Algebra, dass mit einer Kombination von $\uparrow$ ( = NAND) die folgende \textbf{einstelligen} Funktionen dargestellt werden können:
	\begin{enumerate}
		\item $\neg $
		\item $ \id()$
		\item $\top$ (Tautologie)
		\item $\bot$ (Kontradiktion)
	\end{enumerate}
	\item Zeigen sie, dass durch Kombination von $ \uparrow$ die folgenden \textbf{zweistelligen} Wahrheitsfunktionen dargestellt werden können.
	\begin{enumerate}
		\item \& (AND)
		\item $\lor$
		\item $\downarrow$
		\item $\oplus$
	\end{enumerate}
	\item Zeigen sie, dass 
	\begin{enumerate}
		\item \& (AND)
		\item $\lor$
		\item $\oplus$
	\end{enumerate}
	nicht universell ist. D.h. es gibt wenigstens eine bool'sche Funktion, die durch keine Kombination allein der jeweiligen Operationen dargestellt werden kann. 
\end{enumerate}
	
\end{document}