%start preamble
\documentclass[paper=a4,fontsize=11pt]{scrartcl}%kind of doc, font size, paper size

\usepackage{fontspec}
\defaultfontfeatures{Ligatures=TeX}
%\setsansfont{Liberation Sans}
\usepackage{polyglossia}	
\setdefaultlanguage[spelling=new, babelshorthands=true]{german}
\usepackage{csquotes}
		
\usepackage{amsmath}%get math done
\usepackage{amsthm}%get theorems and proofs done
\usepackage{graphicx}%get pictures & graphics done
\graphicspath{{pictures/}}%folder to stash all kind of pictures etc
\usepackage{hyperref}%for links to web
\usepackage{amssymb}%symbolics for math
\usepackage{amsfonts}%extra fonts
\usepackage []{natbib}%citation style
\usepackage{caption}%captions under everything
\usepackage{listings}
\usepackage[titletoc]{appendix}
\numberwithin{equation}{section} 
\usepackage[printonlyused,withpage]{acronym}%how to handle acronyms
\usepackage{float}%for garphics and how to let them floating around in the doc
\usepackage{wrapfig}%making graphics floated by text and not done by minipage
\usepackage{dsfont}
\usepackage{stmaryrd}
\usepackage{geometry}
\usepackage{fancyhdr}
\usepackage{enumitem}
\usepackage[usenames, dvipsnames]{xcolor}



%settings colors for links
\hypersetup{
    colorlinks,
    linkcolor={blue!50!black},
    citecolor={blue},
    urlcolor={blue!80!black}
}

\definecolor{pblue}{rgb}{0.13,0.13,1}
\definecolor{pgreen}{rgb}{0,0.5,0}
\definecolor{pred}{rgb}{0.9,0,0}
\definecolor{pgrey}{rgb}{0.46,0.45,0.48}

\pagestyle{fancy}
\lhead{LGtI -- Übung\\ Wintersemester 2021/22}
\rhead{FB-4 -- AI\\ HTW-Berlin}
%\lfoot{01}
%\cfoot{}
%\fancyfoot[R]{\thepage}
\renewcommand{\headrulewidth}{0.4pt}
%\renewcommand{\footrulewidth}{0.4pt}

\lstdefinestyle{Bash}{
  language=bash,
  showstringspaces=false,
  basicstyle=\small\sffamily,
  numbers=left,
  numberstyle=\tiny,
  numbersep=5pt,
  frame=trlb,
  columns=fullflexible,
  backgroundcolor=\color{gray!20},
  linewidth=0.9\linewidth,
  %xleftmargin=0.5\linewidth
  upquote=true,
  columns=fullflexible,
  literate={*}{{\char42}}1
         {-}{{\char45}}1
}


\newenvironment{solution}
	{
		\color{Blue}
		\textbf{Lösung:}
	}{}

%%here begins the actual document%%
\newcommand{\horrule}[1]{\rule{\linewidth}{#1}} % Create horizontal rule command with 1 argument of height


\DeclareMathOperator{\id}{id}

\title{	
\normalfont \normalsize 
\textsc{Übungsblatt 01}
}
\begin{document}
\vspace*{-1cm}
\begin{center}
\Large{\textbf{Übungsblatt 1}}
\end{center}
\begin{center}\Large{\textbf{Aufgabe A -- Bool'sche Algebra}}\end{center}

\begin{enumerate}
	\item In der Schaltalgebra gelten die folgenden alternativen Absorptionsgesetze:
	\begin{align*}
		(1) & (x \lor \neg y) \land y &= x \land y\\
		(2) & (x \land \neg y) \lor y &= x \lor y
	\end{align*}
	\begin{enumerate}
		\item Beweisen sie die Behauptung, indem sie die nachstehenden Wahrheitstabellen ergänzen:
	\begin{center}
	\begin{table}[ht]
	\centering
\begin{tabular}{|c|c|c|c|c|c|c|c|c|c|}
\hline
 $x$ & $y$ & $\neg x$ & $\neg y$ & $x \lor \neg y$ & $x \land \neg y$ & $(x \lor \neg y) \land y$ & $x \land y$ & $(x \land \neg y) \lor y$ & $x \lor y$ \\ \hline
0 & 0 & 1 & 1 & 1 & 0 & 0 & 0 & 0 & 0 \\ \hline
0 & 1 & 1 & 0 & 0 & 0 & 0 & 0 & 1 & 1 \\ \hline
1 & 0 & 0 & 1 & 1 & 1 & 0 & 0 & 1 & 1 \\ \hline
1 & 1 & 0 & 0 & 1 & 0 & 1 & 1 & 1 & 1 \\ \hline
\end{tabular}
\end{table}
	\end{center}
		\item Führen sie den Beweis erneut, diesmal aber auf algebraische Weise (d.h. durch Umformungen).
		
		\begin{solution}
		
		\begin{align*}
		(1) (x \lor \neg y) \land y &\\
		&= (x \land y) \lor (neg y \land y)\\
		&= (x \land y) \lor 0\\
		&= x \land y
		\end{align*}
		\begin{align*}
		(2) (x \land \neg y) \lor y & \\
		&= (x \lor y) \land (y \lor \neg y)\\
		&= (x \lor y) \land 1\\
		&= x \lor y
		\end{align*}
		\end{solution}
		\item Übertragen sie die Gesetze in die Sprache der Mengenalgebra. Sind sie dort auch gültig?
		
		\begin{solution}
		
		\begin{align*}
		(1) (M \cup \overline{N}) \cap N = M \cap N\\
		(2) (M \cap \overline{N} \cup N = M \cup N
		\end{align*}
		\end{solution}
		\end{enumerate}
	\item Zeigen Sie, dass die booleschen Ausdrücke
	\begin{align*}
		\Phi &= x \land y  \lor \neg ((x \lor \neg y) \land y)\\
		\Psi &= \neg (x \land y) \lor x \lor y
	\end{align*}
	Tautologien sind, indem sie
	\begin{enumerate}
		\item für beide Funktionen eine Wahrheitstafel aufstellen,
		
		\begin{solution}
		\begin{table}[ht]
	\centering
\begin{tabular}{|c|c|c|c|c|c|c|}
\hline
 $x$ & $y$ & $x \land y$ & $x \lor \neg y$ & $(x \land \neg y) \land y$ & $\neg (x \land \neg y) \land y$ & $\varphi$ \\ \hline
0 & 0 & 0 & 1 & 0 & 1 & 1 \\ \hline
0 & 1 & 0 & 0 & 0 & 0 & 1 \\ \hline
1 & 0 & 0 & 1 & 0 & 1 & 1 \\ \hline
1 & 1 & 1 & 1 & 1 & 0 & 1 \\ \hline
\end{tabular}
\end{table}
\begin{table}[ht]
	\centering
\begin{tabular}{|c|c|c|c|c|c|}
\hline
 $x$ & $y$ & $x \land y$ & $\neg(x \land y)$ & $x \lor y$ & $\psi$ \\ \hline
0 & 0 & 0 & 1 & 0 & 1  \\ \hline
0 & 1 & 0 & 1 & 1 & 1 \\ \hline
1 & 0 & 0 & 1 & 1 & 1 \\ \hline
1 & 1 & 1 & 0 & 1 & 1 \\ \hline
\end{tabular}
\end{table}
\end{solution}
		\item den Beweis durch algebraische Umformung führen.
		
		\begin{solution}
		\begin{align*}
		x\land y \lor \neg((x \lor \neg y) \land y) &\\
		&= (x \land y) \lor \neg (x \land y)\\
		&= 1
		\end{align*}
		\begin{align*}
		\neg (x \land y) \lor x \lor y &\\
		&= \neg x \neg y \lor x \lor y \\
		&= (\neg x \lor x) \lor (\neg y \lor y)\\
		&= 1 \lor 1 = 1
		\end{align*}
		\end{solution}
	\end{enumerate}
	\item Bildet das Tripel $( \mathcal{V}, \cdot, +)$ mit
	\begin{align*}
		\mathcal{V} & := \{1,2,3,6\}\\
		\cdot 		& := kgV \text{ (kleinstes gemeinsames Vielfaches)}\\
		+ 			& := ggT \text{(größter gemeinsamer Teiler)}
	\end{align*}	
	eine boolesche Algebra?
	
	\begin{solution}
	\begin{enumerate}
		\item Kommutativegesetz:
		\begin{align*}
		kgV(a,b) &= kgV(b,a)\\
		ggT(a,b) &= ggT(b,a)
		\end{align*}
		\item Distributivgesetz:
		\begin{align*}
		kgV(a, ggT(b,c)) &= ggT(kgV(a,b), kgV(a,c))\\
		ggT(a, kgV(b,c)) &= kgV(ggT(a,b), ggT(a,c))
		\end{align*}
		\item Neutrale Elemente:
		\begin{align*}
		kgV(a,1) &= a\\
		ggT(a,6) &= a
		\end{align*}
		\item Inverse Elemente: Das inverse Element von 1 ist 6 und das inverse Element von 2 ist 3.
		\begin{align*}
		kgV(1,6) &= 6\\
		kgV(2,3) &= 6\\
		ggT(1,6) &= 1\\
		ggT(2,3) &= 1\\
		\end{align*}
	\end{enumerate}
	\end{solution}
	\item Vereinfachen sie die folgenden bool'schen Ausdrücke so weit wie möglich durch die Anwendung der algebraischen Umformungsregeln.
	\begin{enumerate}
		\item $ x_1 \overline{x_2 x_3 x_4} \lor x_1 x_2 \overline{x_3 x_4} \lor x_1 x_2 \overline{x_3} x_4 \lor x_1 \overline{x_2 x_3} x_4 \lor \overline{x_1} x_2 \overline{x_3 x_4} \lor x_1 x_2 x_3 \overline{x_4} \lor \overline{x_1} x_2 x_3 \overline{x_4}$
		\begin{solution}
		\begin{align*}
		& x_1 \overline{x_2 x_3 x_4} \lor x_1 x_2 \overline{x_3 x_4} \lor x_1 x_2 \overline{x_3} x_4 \lor x_1 \overline{x_2 x_3} x_4 \lor \overline{x_1} x_2 \overline{x_3 x_4} \lor \textcolor{red}{ x_1 x_2 x_3 \overline{x_4} \lor \overline{x_1} x_2 x_3 \overline{x_4} } \\
		&= x_1 \overline{x_2} \overline{x_3} \overline{x_4} \lor x_1 x_2 \overline{x_3} \overline{x_4} \lor \textcolor{red}{x_1 x_2 \overline{x_3} x_4 \lor x_1 \overline{x_2} \overline{x_3} x_4 } \lor  \overline{x_1} x_2 \overline{x_3} \overline{x_4} \lor x_2 x_3 \overline{x_4}\\
		&= x_1  \overline{x_2}  \overline{x_3} \overline{x_4} \lor \textcolor{red}{x_1 x_2  \overline{x_3}
x_1 \overline{x_3} } \lor x_2 \overline{x_4} \lor x_1 \overline{x_3} x_4 \lor \overline{x_1} x_2  \overline{x_3} \overline{x_4} \lor x_2 x_3 \overline{x_4}\\
		&= x_1 \overline{x_3} \overline{x_4} \lor x_1 \overline{x_3} x_4 \lor \textcolor{red}{\overline{x_1} x_2 \overline{x_3} \overline{x_4} } \lor x_2 x_3 \overline{x_4}\\
		&= \textcolor{red}{x_1 \overline{x_3} \overline{x_4} \lor x_1 \overline{x_3} x_4} \lor x_2 \overline{x_3} \overline{x_4} \lor x_2 x_3 \overline{x_4}\\
		&= x_1 \overline{x_3} \lor \textcolor{red}{x_2 \overline{x_3} \overline{x_4} \lor x_2 x_3 \overline{x_4}}\\
		&= x_1 \overline{x_3} \lor x_2 \overline{x_4}
		\end{align*}
		\end{solution}
	\end{enumerate}
	\item Zeigen oder widerlegen sie die folgende Beziehung zwischen den Operatoren $\Leftrightarrow$ und $\not\Leftrightarrow$:
	\begin{enumerate}
		\item $x \Leftrightarrow y \Leftrightarrow z = x \not\Leftrightarrow y \not\Leftrightarrow z$
		
		\begin{solution}
		
		Die Beziehung ist gültig!
		\begin{align*}
		&~ x \Leftrightarrow y \Leftrightarrow z \\
		&= (x \Leftrightarrow y) \Leftrightarrow z \\
		&= (\overline{\overline{x \Leftrightarrow y}}) \Leftrightarrow z\\
		&= (\overline{x \Leftrightarrow y}) \not \Leftrightarrow z\\
		&= (x \not \Leftrightarrow y) \not \Leftrightarrow z\\
		&= x \not \Leftrightarrow y \not \Leftrightarrow z\\
		\end{align*}
		\textbf{Die Äquivalenz gilt nur, wenn die Anzahl der Operanden ungerade ist}
		\end{solution}
	\end{enumerate}
	\item Zeigen sie, dass die folgenden Varianten des Distributivgesetzes für $\Leftrightarrow$ und $\neg \Leftrightarrow$ falsch sind:
	\begin{enumerate}[resume]
		\item $(x \lor z) \neg \Leftrightarrow (y \lor z) = (x \neg \Leftrightarrow y) \lor z$
		
		\begin{solution}

		Gegenbeispiel: $x= 0,y= 0,z= 1$
		\begin{align*}
		(x \lor y) \not \Leftrightarrow (y \lor z) &= (0 \lor 1) \not \Leftrightarrow (0 \lor 1)\\
		&= 1 \not \Leftrightarrow 1\\
		(x \not \Leftrightarrow y) \lor z &= (0 \not \Leftrightarrow 0) \land 1\\
		&= 1
		\end{align*}
		\end{solution}
		\item $(x \land z) \Leftrightarrow (y \land z) = (x \Leftrightarrow y) \land z$
		
		\begin{solution}
		
		Gegenbeispiel: $x= 0,y= 0,z= 0$
		\begin{align*}
		(x \land z) \Leftrightarrow (y \land z) &= (0 \land 0) \Leftrightarrow (0 \land 0)\\
		&= 0 \Leftrightarrow 0\\
		&= 1\\
		(x \Leftrightarrow y) \land z &= (0 \not \Leftrightarrow 0) \land 0\\
		&= 0
		\end{align*}
		\end{solution}
	\end{enumerate}
	\item Nachstehend sind die erweiterten De Morgan'schen Regeln aufgeführt.
	\begin{enumerate}
		\item $\overline{x_1 \land x_2 \land \ldots \land x_n} = \overline{x_1} \lor \overline{x_2} \lor \ldots \lor \overline{x_n}$
		\item $\overline{x_1 \lor x_2 \lor \ldots \lor x_n} = \overline{x_1} \land \overline{x_2} \land \ldots \land \overline{x_n}$
		
		\begin{solution}
		
		Induktionsanfang (IA): Für den Basisfall $n_2$ fallen die traditionelle und die erweiterte De Morgansche Regel zusammen. Die Aussage ist damit für den Fall $n= 2$ gültig.\\
		Induktionsvoraussetzung (IV): Für ein gewisses $n$ gelte:
		\begin{align*}
		(\overline{x_1 \land x_2 \land \ldots \land x_n}) &= \overline{x_1} \lor \ldots \overline{x_n}
		\end{align*}
		Induktionsschritt (IS):
		\begin{align*}
		(\overline{x_1 \land x_2 \land \ldots \land x_n}) &=  ((\overline{x_1 \land x_2 \land \ldots \land x_n) \land x_{n+1}})\\
		&= ((\overline{x_1 \land x_2 \land \ldots \land x_n)} \lor \overline{x_{n+1}}\\
		&= ((\overline{x_1} \lor \overline{x_2} \lor \ldots \lor \overline{x_n}) \lor  \overline{x_{n+1}}) \\
		&= \overline{x_1} \lor \overline{x_2} \lor \ldots \lor \overline{x_n}
		\end{align*}
		\end{solution}
	\end{enumerate}
	\item Gegeben seien die folgenden drei bool'schen Funktionen:
	\begin{enumerate}
		\item $\varphi_1 := (x \Rightarrow y) \Rightarrow z$
		\begin{solution}
		
		\begin{align*}
		&~ (x \Rightarrow y) \Rightarrow z\\
		&= \overline{x \Rightarrow y} \lor z\\
		&= \overline{\overline{x} \lor y} \lor z\\
		&= (x \land \overline{y}) \lor z\\
		&= (x \lor z) \land (\overline{y} \lor z)\\
		&= \overline{\overline{(x \lor z) \land (\overline{y} \lor z)}}\\
		&= \overline{\overline{(x \lor z)} \lor \overline{(\overline{y} \lor z)}} \\
		&= \overline{\overline{(x \lor z)} \lor \overline{(\overline{y \lor y} \lor z})}
		\end{align*}
		\end{solution}
		\item $\varphi_2 := x \Rightarrow (y \Rightarrow z)$
		
		\begin{solution}
		\begin{align*}
		&~ x \Rightarrow (y \Rightarrow z)\\
		&= \overline{x} \lor (y \Rightarrow z)\\
		&= \overline{x} \lor (\overline{y} \lor z)\\
		&= (\overline{x} \lor \overline{y}) \lor z\\
		&= \overline{\overline{(\overline{x} \lor \overline{y}) \lor z}}\\
		&= \overline{(\overline{x \lor \overline{y})})\land \overline{z}}\\
		&= \overline{(\overline{\overline{(x \land y)}}\land \overline{z}}\\
		&= \overline{\overline{\overline{(x \land y)} \land \overline{(x \land y)}} \land \overline{z \land z}}
		\end{align*}
		\end{solution}
		\item $\varphi_3 := \overline{x \land y} \lor \overline{x \land \overline{z}}$
		
		\begin{solution}
		\begin{align*}
		&~ \overline{x \land y} \lor \overline{x \land \overline{z}}\\
		&= \overline{x} \lor \overline{y} \lor \overline{x} \lor z\\
		&= \overline{x} \lor \overline{y} \lor z\\
		&= \overline{x} \lor (\overline{y} \lor z)\\
		&= \overline{x} \lor (y \Rightarrow z)\\
		&= x \Rightarrow (y \Rightarrow z)
		\end{align*}
		\end{solution}
	\end{enumerate}
	Stellen sie $\varphi_1$ unter ausschließlicher Verwendung der NOR-Funktion, $\varphi_2$ unter ausschließlicher Verwendung der NAND-Funktion und $\varphi_3$ unter ausschließlicher Verwendung der Implikation dar.
	\item Zeigen sie unter Anwendung der Regeln der bool'schen Algebra, dass mit einer Kombination von $\uparrow$ ( = NAND) die folgende \textbf{einstelligen} Funktionen dargestellt werden können:
	\begin{enumerate}
		\item $\neg $
		\item $ \id()$
		\item $\top$ (Tautologie)
		\item $\bot$ (Kontradiktion)
	\end{enumerate}
	
	\begin{solution}
	\begin{enumerate}
	\item $\neg$
	\begin{proof}
$\neg$ kann durch kombiniertes Anwenden von $\uparrow$ dargestellt werden\\
$\neg A \Leftrightarrow \neg (A +A) \Leftrightarrow A \uparrow A $ ,da $A + A \Leftrightarrow \id(A)$
\end{proof}
\item $id$

\begin{proof}
$\id$ kann durch kombiniertes Anwenden von $\uparrow$ dargestellt werden\\
$\id \Leftrightarrow \neg(\neg A) \Leftrightarrow A$ \\
aus a) folgt $\neg(A + A) \Leftrightarrow (A \uparrow A) \Leftrightarrow \neg A$ , daraus ergibt
$(A \uparrow A) \uparrow (A \uparrow A) \Leftrightarrow \neg(\neg(A)) \Leftrightarrow A$
\end{proof}
\item $\top$

\begin{proof}
$\top$ kann durch kombiniertes Anwenden von $\uparrow$ dargestellt werden\\
Wir verwenden einfach a) $\neg$ und c) $\top$
\end{proof}
\item $\bot$
\begin{proof}
$\bot$ kann durch kombiniertes Anwenden von $\uparrow$ dargestellt werden\\
Idee: Kontradiktion als Negation der Tautologie, die Tautologie wird mittels $\neg$ invertiert.\\
$ \Leftrightarrow \neg \top \Leftrightarrow \bot$  da die Kontradiktion die Inversion der Tautologie ist - also unsere Voraussetzung\\
$ \Leftrightarrow (\top \uparrow \top)$ die Negation aus a) und die Tautologie aus c)\\
$\bot = \top \uparrow \top = ((A \uparrow A) \uparrow A) \uparrow ((A \uparrow A) \uparrow A)$
\end{proof}
\end{enumerate}
	\end{solution}
	\item Zeigen sie, dass durch Kombination von $ \uparrow$ die folgenden \textbf{zweistelligen} Wahrheitsfunktionen dargestellt werden können.
	\begin{enumerate}
		\item \& (AND)
		\item $\lor$
		\item $\downarrow$
		\item $\oplus$
	\end{enumerate}
	
	\begin{solution}
	
	\begin{enumerate}
\item $\&$
\begin{proof}
$\&$ kann durch kombiniertes Anwenden von $\uparrow$ dargestellt werden\\
Idee: $ \neg (A \& B) \Leftrightarrow A \uparrow B $ eine doppelte Negation hebt sich auf!\\
$ \neg (\neg (A \& B) \& \neg (A \& B))$\\
$ \Leftrightarrow \neg (A \& B) \uparrow \neg (A \& B)$\\
$ \Leftrightarrow (A \uparrow B) \uparrow (A \uparrow B)$
\end{proof}
\item $\lor$
\begin{proof}
$\lor$ kann durch kombiniertes Anwenden von $\uparrow$ dargestellt werden\\
Idee: das logische OR ist assoziativ ($a+(b+c)=(a+b)+c$) und \textit{kommunikativ} $(a + b = b +a)$ \\ 
nach de Morgan gilt: $\neg(A \& B) \Leftrightarrow \neg A \lor \neg B$ bzw. $\neg (A \lor B) \Leftrightarrow \neg (A) \& \neg(B) $\\
$A \lor B \Leftrightarrow \neg (\neg A \& \neg B)$ ,wobei $\neg A \Leftrightarrow A \uparrow A$ ist (s. o)\\
$ \Leftrightarrow \neg((A \uparrow A) \& (B \uparrow B))$ umformen\\
$ \Leftrightarrow (A \uparrow A) \uparrow (B \uparrow B)$ 
\end{proof}
\item $\downarrow$

\begin{proof}
$\downarrow$ kann durch kombiniertes Anwenden von $\uparrow$ dargestellt werden\\
Idee: nach anwenden von de Morgan: $A \downarrow B \Leftrightarrow \neg (A \lor B) \Leftrightarrow \neg A \& \neg B$\\
Da wir schon den binären Operator $\lor$ bewiesen haben wenden wir nun noch den unären Operator $\neg$ darauf an. Somit ergibt sich:\\
$((A \uparrow A) \uparrow (B \uparrow B)) \uparrow ((A \uparrow A) \uparrow (B \uparrow B))$
\end{proof}
\item $\oplus$ 
\begin{proof}
$\oplus$ kann durch kombiniertes Anwenden von $\uparrow$ dargestellt werden\\
Idee: Wir brauchen einen ausschließendes Oder, M.a.W nur die Tupel TRUE FALSE, FALSE TRUE ergeben wahr. 
Demnach suchen wir eine Operation, die nur bei der Verknüpfung von A und B wahr liefert, wenn ein Eingang wahr ist und der zweite falsch. Die kann durch $(\alpha \& \neg \beta)$ realisiert werden. Verknüpfen wir so unsere beiden Eingaben mittels $\lor$ werden nur die Eingaben TRUE liefern, die mindestens ein TRUE-Wert haben.\\
$(A \oplus B) \Leftrightarrow (A \& \neg B) \lor (\neg A \& B)$\\
$\Leftrightarrow (A \& (B \uparrow B)) \lor ((A \uparrow A) \& B) $ Negation von B, wie oben\\
$\Leftrightarrow [(A \uparrow (B \uparrow B))\uparrow (A \uparrow (B \uparrow B))] $
$ \lor [ ((A \uparrow A) \uparrow B) \uparrow ((A \uparrow A) \uparrow B)] $ \\
$\Leftrightarrow [(A \uparrow (B \uparrow B))\uparrow (A \uparrow (B \uparrow B)) \uparrow (A \uparrow (B \uparrow B))\uparrow (A \uparrow (B \uparrow B))]$ \\
$ \uparrow [ ((A \uparrow A) \uparrow B) \uparrow ((A \uparrow A) \uparrow B) \uparrow ((A \uparrow A) \uparrow B) \uparrow ((A \uparrow A) \uparrow B)] $\\
$\Leftrightarrow ((A \uparrow B) \uparrow B) \uparrow ((B \uparrow A) \uparrow A)$
\end{proof}
\end{enumerate}
	\end{solution}
	\item Zeigen sie, dass 
	\begin{enumerate}
		\item \& (AND)
		\item $\lor$
		\item $\oplus$
	\end{enumerate}
	nicht universell ist. D.h. es gibt wenigstens eine bool'sche Funktion, die durch keine Kombination allein der jeweiligen Operationen dargestellt werden kann. 
	
	\begin{solution}
	
	\begin{enumerate}
\item $\&$
\begin{proof}
Annahme: Der Operator $\uparrow$ sei mittels \& darstellbar.\\
$(A \uparrow B) \Leftrightarrow A \& B \Leftrightarrow (A \& B) \& (A \& B) \Leftrightarrow (A \& A) \& (B \& B) \Leftrightarrow A \& B \lightning$ \\ 
Da sich der \& Operator neutral verhält ist es nicht möglich zum $\uparrow$ zu kommen. M.a.W. es gibt keinen Weg zum NAND unter ausschließlicher Verwendung von \&.\\
$(A \uparrow B) \nLeftrightarrow (A \& B)$ 
\end{proof}
\item $\lor$
\begin{proof}
Annahme: Wir können den NOR Operator durch $\lor$ darstellen.\\
$(A \downarrow B) \Leftrightarrow (A \uparrow A) \uparrow (B \uparrow B)$\\
$A \uparrow A \Leftrightarrow \neg A$ Da sich die Negation nicht aus einer auf sich selbt neutralen Operation herleitbar, ist auch das NOR nicht ableitbar aus OR und somit nicht universell.\\
\end{proof}
\item $\oplus$ 
\begin{proof}
Es wird versucht das $\neg$ zu bilden.\\
$\neg A \Leftrightarrow A \oplus A = \bot \lightning$ ergibt die Kontradiktion.\\
$\neg A \Leftrightarrow \bot \oplus \bot = \bot \lightning$ verhält also sich neutral.\\
$\neg A \Leftrightarrow \bot \oplus A = A \lightning $ verhält sich ebenfalls neutral.\\
$\neg A \Leftrightarrow A \oplus \bot = A \lightning $ verhält sich ebenfalls neutral.\\
Da wir keine Negation als unären Operator aus XOR erzeugen können, ist dieser nicht universell.
\end{proof}
\end{enumerate}
	\end{solution}
\end{enumerate}
	
\end{document}