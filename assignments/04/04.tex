%start preamble
\documentclass[paper=a4,fontsize=11pt]{scrartcl}%kind of doc, font size, paper size

\usepackage{fontspec}
\defaultfontfeatures{Ligatures=TeX}
%\setsansfont{Liberation Sans}
\usepackage{polyglossia}	
\setdefaultlanguage[spelling=new, babelshorthands=true]{german}
\usepackage{csquotes}
		
\usepackage{amsmath}%get math done
\usepackage{amsthm}%get theorems and proofs done
\usepackage{graphicx}%get pictures & graphics done
\graphicspath{{pictures/}}%folder to stash all kind of pictures etc
\usepackage{hyperref}%for links to web
\usepackage{amssymb}%symbolics for math
\usepackage{amsfonts}%extra fonts
\usepackage []{natbib}%citation style
\usepackage{caption}%captions under everything
\usepackage{listings}
\usepackage[titletoc]{appendix}
\numberwithin{equation}{section} 
\usepackage[printonlyused,withpage]{acronym}%how to handle acronyms
\usepackage{float}%for garphics and how to let them floating around in the doc
\usepackage{cclicenses}%license!
\usepackage{xcolor}%nicer colors, here used for links
\usepackage{wrapfig}%making graphics floated by text and not done by minipage
\usepackage{dsfont}
\usepackage{fancyhdr}
\usepackage{enumitem}



%settings colors for links
\hypersetup{
    colorlinks,
    linkcolor={blue!50!black},
    citecolor={blue},
    urlcolor={blue!80!black}
}

\definecolor{pblue}{rgb}{0.13,0.13,1}
\definecolor{pgreen}{rgb}{0,0.5,0}
\definecolor{pred}{rgb}{0.9,0,0}
\definecolor{pgrey}{rgb}{0.46,0.45,0.48}

\pagestyle{fancy}
\lhead{LtGdI -- Übung\\ Wintersemester 2021/22}
\rhead{FB-4 -- AI\\ HTW-Berlin}
%\lfoot{01}
%\cfoot{}
%\fancyfoot[R]{\thepage}
\renewcommand{\headrulewidth}{0.4pt}
%\renewcommand{\footrulewidth}{0.4pt}

\lstdefinestyle{Bash}{
  language=bash,
  showstringspaces=false,
  basicstyle=\small\sffamily,
  numbers=left,
  numberstyle=\tiny,
  numbersep=5pt,
  frame=trlb,
  columns=fullflexible,
  backgroundcolor=\color{gray!20},
  linewidth=0.9\linewidth,
  %xleftmargin=0.5\linewidth
  upquote=true,
  columns=fullflexible,
  literate={*}{{\char42}}1
         {-}{{\char45}}1
}

\lstdefinelanguage
   [x86]{Assembler}
   [x86masm]{Assembler} % based on the "x86masm" dialect
   % with these extra keywords:
   {morekeywords={CDQE,CQO,CMPSQ,CMPXCHG16B,JRCXZ,LODSQ,MOVSXD, %
                  POPFQ,PUSHFQ,SCASQ,STOSQ,IRETQ,RDTSCP,SWAPGS, %
                  eax,edx,ecx,ebx,esi,edi,esp,ebp, %
                  e8,e8d,e8w,e8b,e9,e9d,e9w,e9b, %
                  e10,e10d,e10w,e10b,e11,e11d,e11w,e11b, %
                  e12,e12d,e12w,e12b,e13,e13d,e13w,e13b, %
                  e14,e14d,e14w,e14b,e15,e15d,e15w,e15b}} %

\lstset{language=[x86]Assembler}


\lstset{language=C,
	basicstyle=\ttfamily,
   	keywordstyle=\color{blue}\ttfamily,
    stringstyle=\color{red}\ttfamily,
    commentstyle=\color{cyan}\ttfamily,
    morecomment=[l][\color{magenta}]{\#}
    showstringspaces=false,
  	basicstyle=\small\sffamily,
  	numbers=left,
  	numberstyle=\tiny,
  	numbersep=5pt,
  	frame=trlb,
  	upquote=true,
  	columns=fullflexible,
  	backgroundcolor=\color{gray!20},
  	%linewidth=0.9\linewidth,
  	literate=*{*}{\normalfont{*}}1,
}


\newcommand{\specialcell}[2][c]{%
  \begin{tabular}[#1]{@{}c@{}}#2\end{tabular}}

%%here begins the actual document%%
\newcommand{\horrule}[1]{\rule{\linewidth}{#1}} % Create horizontal rule command with 1 argument of height


\DeclareMathOperator{\id}{id}

\title{	
\normalfont \normalsize 
\textsc{Übungsblatt 01}
}
\begin{document}
\vspace*{-1cm}
\begin{center}
\Large{\textbf{Übungsblatt 4}}
\end{center}

\begin{center}\Large{\textbf{Aufgabe A -- Zahlendarstellung}}\end{center}
\begin{enumerate}
	\item Bestimmen Sie nachvollziehbar (d. h. mit Zwischenschritten) die Darstellung der gegebenen natürlichen Zahlen in der jeweils angegebenen Basis:
	\begin{enumerate}
		\item $5453_6 = (\dots)_2$
		\item $72_{10} = (\dots)_3$
		\item $654_7 = (\dots)_9$
		\item $17HAI_{26}=(\dots)_{36}$
	\end{enumerate}
	\item In dieser Aufgabe sollen sie in Java eine Funktion implementieren, die ganze Zahlen des Dezimalsystems als Zahlen des Dualsystems darstellt. Gehen Sie dabei in mehreren Schritten vor:
	\begin{enumerate}
		\item Implementieren  Sie  eine  Funktion \textit{ntobasetwo(int n, int c)},  die  eine  natürliche  Zahl $n\in \mathbb{N}$ in eine Binärzahl der Länge $c$ umwandelt. Als Rückgabewert wird ein Vektor $b$ der Länge $c$ erwartet, sodass
$$
    n = \sum_{i=1}^c b_i \cdot 2^{i-1}
$$
sowie $b_i \in \{0,1\}$ für alle $i\in\{1, \dots ,c\}$ gilt.
	\end{enumerate}
	Für den Fall, dass $n$ nicht als Binärzahl der Länge $c$ dargestellt werden kann, soll Ihr Programm das Ergebnis entsprechend abschneiden. Sprich, falls $n = \sum_{i=1}^m b_i \cdot 2^{i-1}$ mit $m > c$ gilt, sollen nur die $b_i$ mit $i\in \{1,\dots , c\}$ zurückgegeben werden.\\

\textbf{Hinweis:} Sie dürfen hier und im Folgenden frei wählen, ob Sie den Rückgabewert als numerischen oder als logischen Vektor realisieren.
	\item Implementieren Sie eine Funktion \textit{complement(b)}, die das Zweierkomplement einer Binärzahl entsprechend der Vorlesung berechnet. Dabei wird als Eingabe ein Vektor $b$ erwartet mit $b_i \in \{0,1\}$. Der Rückgabewert soll auch ein Vektor $\hat b$ mit $\hat b_i \in \{0,1\}$ sein, sodass $b$ und $\hat b$ dieselbe Länge haben.
\end{enumerate}

\end{document}