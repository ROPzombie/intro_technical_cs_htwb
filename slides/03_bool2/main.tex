\documentclass[12pt%
%,draft%
,aspectratio=169%
]{beamer}
%
\usepackage{fontspec}
\defaultfontfeatures{Ligatures=TeX}
%\setsansfont{Liberation Sans}
\usepackage{polyglossia}
\setdefaultlanguage{ngerman}
% Alternative template for talks of the Freie Universität Berlin.
% Created by Leonard R. König, <leonard.koenig@fu-berlin.de> following the
% guidelines on www.fu-berlin.de/cd
%
% (c) Leonard König, CC BY 4.0
%
% This template was written against UTF-8 capable LaTeX engines, specifically
% LuaLaTeX.

% Trying to get rather close to the ppt/odp template:
%  http://www.fu-berlin.de/sites/cd/downloads_container/PowerPoint_Praesentation_Anleitung.pdf

%%% font styles
\setbeamerfont{frametitle}{series=\bfseries}
\setbeamerfont{footline}{series=\bfseries}
\setbeamerfont{headline}{series=\bfseries}
\setbeamerfont{alerted text}{series=\bfseries}
%%%

% colordefs
\definecolor{fu_darkblue}{RGB}{0,51,102}
\definecolor{fu_seablue}{RGB}{0,102,204}
\definecolor{fu_lightblue}{RGB}{204,214,224}
\definecolor{fu_green}{RGB}{153,204,0}
\definecolor{fu_lightgrey}{RGB}{128,128,128}
\definecolor{fu_grey}{RGB}{95,95,95}
%
\definecolor{fu_red}{RGB}{204, 0, 0} % red text (used by \alert)
%%% end colordefs

%%% colors
\setbeamercolor*{title}{fg=fu_darkblue}
\setbeamercolor*{subtitle}{fg=fu_seablue}
\setbeamercolor*{frametitle}{fg=fu_darkblue}
\setbeamercolor*{footline}{fg=fu_grey,bg=fu_lightblue}
\setbeamercolor*{headline}{fg=fu_grey}

\setbeamercolor*{normal text}{fg=black}
\setbeamercolor*{alerted text}{fg=fu_red}
\setbeamercolor*{example text}{fg=fu_green}
\setbeamercolor*{structure}{fg=fu_darkblue}

\setbeamercolor*{block title}{fg=white,bg=black!50}
\setbeamercolor*{block title alerted}{fg=white,bg=black!50}
\setbeamercolor*{block title example}{fg=white,bg=black!50}

\setbeamercolor*{block body}{bg=black!10}
\setbeamercolor*{block body alerted}{bg=black!10}
\setbeamercolor*{block body example}{bg=black!10}

\setbeamercolor{bibliography entry author}{fg=fu_darkblue}

\setbeamercolor{item}{fg=fu_darkblue}
\setbeamercolor{navigation symbols}{fg=fu_lightgrey,bg=fu_grey}
%%% end colors

%%% title page
% Display logo (if exists) and right next to it, put our title + subtitle
\defbeamertemplate*{title page}{fu_titlepage}
{%
	\hskip .3\textheight
	\begin{minipage}[.4\textheight]{\textwidth}
		\begin{minipage}[.4\textheight]{0.25\textwidth}
			\inserttitlegraphic
		\end{minipage}%
		\begin{minipage}[.4\textheight]{0.75\textwidth}
			\begin{beamercolorbox}{title}
				\usebeamerfont{title}\inserttitle\par%
			\end{beamercolorbox}
			\vfill
			\ifx\insertsubtitle
				\@empty%
			\else
				\begin{beamercolorbox}{subtitle}
					\usebeamerfont{subtitle}\insertsubtitle\par
				\end{beamercolorbox}
			\fi
		\end{minipage}
	\end{minipage}%
	\hskip .3\textheight
}
%%% end title page

%%% headline
% display title, author and institute on the left;
% logo on the right.
\newcommand{\headlinetext}
{%
	\inserttitle\\[0.3em]%
	\insertauthor, %
	\insertshortinstitute
}
\newlength{\headlinewidth}
\setlength{\headlinewidth}{\paperwidth}
\addtolength{\headlinewidth}{-2\marginparsep}
\setbeamertemplate{headline}
{%
	\begin{beamercolorbox}[wd=\paperwidth]{headline}%
		\vskip5pt
		{\hspace*{\marginparsep}}%
		\parbox{.5\headlinewidth}
		{%
			\usebeamertemplate{title in head/foot}%
			\headlinetext%
		}%
		\begin{minipage}{.5\headlinewidth}%
			\hfill\usebeamertemplate*{logo}
		\end{minipage}%
		{\hspace*{\marginparsep}}%
	\end{beamercolorbox}%
}
%%% end headline

%%% footline
% title + date on the left, frame number on the right
\newcommand{\footlinetext}
{%
	\usebeamerfont{shorttitle}\insertshorttitle, %
	\usebeamerfont{shortdate}\insertshortdate
}
\setbeamertemplate{footline}
{%
	\begin{beamercolorbox}{footline}
		\vskip2pt
		\hspace{\marginparsep}%
		\footlinetext\hfill%
		\insertframenumber%
		\hspace{\marginparsep}
		\vskip2pt
	\end{beamercolorbox}%
}
%%% end footline

% don't use default templates for sidebars
\setbeamertemplate{sidebar right}{}
\setbeamertemplate{sidebar left}{}
\setbeamertemplate{title page}[fu_titlepage]
\usepackage{amsmath}
\usepackage{amsfonts}
\usepackage{amssymb}
\usepackage{graphicx}
\usepackage{algorithm}
\usepackage[noend]{algpseudocode}
%\usepackage{algorithmic}
\usepackage{tikz}
\usetikzlibrary{arrows,shapes,automata,petri,positioning,calc}
\usepackage{graphicx}
\usepackage{subfig}
\usepackage{pgfplots}
\usepackage{venndiagram}
\usepackage{ stmaryrd }



\pgfplotsset{
    standard/.style={%Axis format configuration
        axis x line=middle,
        axis y line=middle,
        enlarge x limits=0.15,
        enlarge y limits=0.15,
        every axis x label/.style={at={(current axis.right of origin)},anchor=north west},
        every axis y label/.style={at={(current axis.above origin)},anchor=north east},
        every axis plot post/.style={mark options={fill=white}}
        }
    }


\author{Benjamin Tröster}
\title[Bool'sche Algebra]{Bool'sche Algebra}
%\subtitle[Markov Models]{...}
%\pgfdeclareimage{titlegraphic}{../res/dwarf_logo2.png}
%\titlegraphic{\pgfuseimage{titlegraphic}}
%\date{}
%\subject{}
%
% FU settings
\institute[HTW Berlin]{Hochschule für Technik und Wirtschaft Berlin}
%\pgfdeclareimage[height=0.9cm]{logo}{../res/dwarf_logo}
%\logo{\pgfuseimage{logo}}
%
\usepackage[
backend=biber,
citestyle=alphabetic,bibstyle=authoryear
]{biblatex}
\addbibresource{sources.bib}


\begin{document}

\begin{frame}
\titlepage
\end{frame}

\begin{frame}{Fahrplan}
\tableofcontents[hideothersubsections]
\end{frame}

\section{Recap}
\begin{frame}{Aussagenlogik}
\begin{definition}[Aussagenlogik]
\textbf{Aussagenlogik}, als Teilgebiet der Logik, befasst sich mit Aussagen und der Verknüpfung von Aussagen mittels \textit{Junktoren}.
\end{definition}
\begin{itemize}
	\item Junktoren sind logische Verknüpfungen
	\item Klassische Junktoren:
	\begin{itemize}
		\item Negation $\neg P$ 
		\item Implikation/Subjunktion/Konditional $P\Rightarrow Q$
		\item Äquivalenz/Bikonditional/Bisubjunktion $P\Leftrightarrow Q$
		\item Konjunktion $P\land Q$
		\item Disjunktion $P\lor Q$
	\end{itemize}
\end{itemize}
\cite{rautenberg2002einfuhrung}
\end{frame}

\begin{frame}{Bool'sche Algebra nach Huntington (\textbf{Wichtig!})}
\begin{definition}
Die bool'sche Algebra nach Huntington ist definiert als Menge $\mathcal{V}: \{0,1\}$ mit den Verknüpfungen $\cdot (\land), + (\lor)$, sodass $\mathcal{V} \times \mathcal{V} \to \mathcal{V}$, also $\{0,1\} \times \{0,1\} \to \{0,1\}$. 
\end{definition}
\begin{itemize}
	\item Kommutativgesetze (K): $a \cdot b = b \cdot a$ bzw. $a + b = b + a$
	\item Distributivgesetze (D): $a \cdot (b + c) = (a \cdot b) + (a \cdot c)$ bzw. $a + (b \cdot c) = (a + b) \cdot (a + c)$
	\item Neutrale Elemente (N): $ \exists e, n \in \mathcal{V}$ mit  $a \cdot e = a$ und $a + n = a$
	\item Inverse Elemente (I): $\forall a \in \mathcal{V}$ existiert ein $a'$ mit $a \cdot a'= n$ und $a + a' = e$
\end{itemize}
Übernommen von \cite{barnett2013boolean} bzw. \cite{hoffmann2020grundlagen}
\end{frame}

\begin{frame}{Darstellungen \& Bool'sche Funktionen}
\begin{itemize}
	\item Wahrheitstabelle
	\begin{center}
\begin{table}[]
\begin{tabular}{|c|c|c|ll}
\cline{1-3}
$a$ & $b$ & $a \Rightarrow b$ &  &  \\ \cline{1-3}
0 & 0 & 1 &  &  \\ \cline{1-3}
0 & 1 & 1 &  &  \\ \cline{1-3}
1 & 0 & 0 &  &  \\ \cline{1-3}
1 & 1 & 1 &  &  \\ \cline{1-3}
\end{tabular}
\end{table}
\end{center}
	\item 
	\resizebox{5cm}{!}{%
\begin{tikzpicture}[>=stealth',auto,shorten >=1pt]
    \node [place] (tlor) {$\lor$};
    
    \node [place] (llor) [below left=of tlor] {$\land$};
	\node [place] (rlor) [below right=of tlor] {$\land$};
	\node [place] (b0) [below left=of llor] {$0$};
	\node [place] (bx) [below =of llor] {$x$};
	
	\node [place] (b1) [below =of rlor] {$1$};
	\node [place] (bx2) [below right=of rlor] {$x$};

	\path[-] (tlor) edge node [left] {}  (llor);
	\path[-] (tlor) edge node [left] {}  (rlor);
	\path[-] (llor) edge node [left] {}  (b0);
	\path[-] (llor) edge node [left] {}  (bx);
	\path[-] (rlor) edge node [left] {}  (b1);
	\path[-] (rlor) edge node [left] {}  (bx2);

\end{tikzpicture}
}
	\item Algebraische Darstellung: $y = ((0 \land x) \lor (1 \lor x))$
\end{itemize}
\end{frame}

\begin{frame}{Notation und Operatorenbindung}
\begin{itemize}
	\item Syntactic Sugar (Ableitungen aus Basisverknüpfungen)
	\begin{itemize}
		\item $(a \Rightarrow b)$ für $(\neg a \lor b)$ -- Implikation
		\item $(a \Leftarrow b)$ für $(b \Rightarrow a)$ -- Inversion der Implikation
		\item $(a \Leftrightarrow b)$ für $(a \Rightarrow b) \land (a \Leftarrow b)$ -- Äquivalenz
		\item $(a \oplus b)  für ¬(a \Leftrightarrow b$ -- Antivalenz oder Exklusiv-ODER/XOR
		\item $\neg(a \lor b)$ -- NOR
		\item $\neg (a \land b)$ -- NAND
	\end{itemize}
	\item Bindung der Operatoren 
	\begin{itemize}
		\item $\land$ bindet stärker als $\lor$
		\item $\neg$ bindet stärker als $\land$
	\end{itemize}
	\item Klammerung
	\begin{itemize}
		\item Gleiche Verknüpfungen: linksassoziativ zusammengefasst
	\end{itemize}
\end{itemize}	 
\end{frame}

\begin{frame}{Beispiel}
\begin{align*}
Y &= (A \lor B) \land (\neg A \lor B) \land (A \lor \neg B)
\end{align*}
\end{frame}

\begin{frame}{Beispiel}
\begin{align*}
Y &= (A \lor B) \land (\neg A \lor B) \land (A \lor \neg B)\\
Y &= (\neg a \land \neg b) \lor (a \land b)
\end{align*}
\end{frame}


\section{Einleitung}
\begin{frame}{}
\begin{itemize}
	\item Erfüllbarkeit \& Äquivalenz
	\item De Morgan Regeln
	\item Universelle Operatoren
	\item Beweisstrategien \& Induktion -- Strukturelle Induktion
	\item Dualitätsprinzip
	\item Normalformen
	\item Bitweise logische Operationen, Bit-Maskierung
	\item Einführung Logikgatter
\end{itemize}
\end{frame}

\section{Erfüllbarkeit \& Äquivalenz}
\begin{frame}{Erfüllbarkeit}
	\begin{definition}[Erfüllbarkeit]
		Sei $\varphi$ ein beliebiger boolescher Ausdruck. $\varphi$ heißt
		\begin{itemize}
			\item erfüllbar, wenn es Werte $x_1, \ldots, x_n$ gibt, mit $\varphi (x_1, \ldots, x_n ) = 1$.
			\item widerlegbar, wenn es Werte $x_1, \ldots, x_n$ gibt, mit $\varphi (x_1, \ldots, x_n) = 0$.
			\item unerfüllbar, wenn $\varphi (x_1 ,\ldots , x_n )$ immer gleich $0$ ist.
			\item allgemeingültig, wenn wenn $\varphi (x_1 ,\ldots , x_n )$ immer gleich $1$ ist.
		\end{itemize}
		Einen allgemeingültigen Ausdruck bezeichnen wir auch als \textbf{Tautologie}.
	\end{definition}
\end{frame}

\begin{frame}{Erfüllbarkeit/Unerfüllbar/Allgemeingültig}
\begin{itemize}
	\item $\phi = \neg x$
	\item $\phi = x \land \neg x$
	\item $\neg (x \land \neg x)$
\end{itemize}
\end{frame}

\begin{frame}{Äquivalenz}
\begin{definition}[Äquivalenz]
Zwei bool'sche Ausdrücke $\varphi$ und $\psi$ sind äquivalent, falls sie dieselbe Funktion repräsentieren. In anderen Worten: $\varphi und \psi$ sind genau dann äquivalent, wenn für alle Variablenbelegungen $x_1 , \ldots, x_n$ die folgende Beziehung gilt:
$$ \varphi (x_1,\ldots, x_n) = \psi(x_1, \ldots, x_n)$$
\end{definition}
D.h. Zwei bool'sche Ausdrücke $\phi$ und $\psi$ sind genau dann äquivalent, wenn der Ausdruck $\varphi \Leftrightarrow \psi$ eine Tautologie ist.\\
Mithilfe von Wahrheitstafeln, algebraischer Umformung oder durch erzeugen einer Normalform können wir die Äquivalenz feststellen.
\end{frame}

\section{Beweisstrategien}
\begin{frame}{Beweisstrategien}
\begin{itemize}
	\item Direkter Beweis
	\begin{itemize}
		\item Annahme: $A$ ist allgemeingültig, durch richtiges Schließen: $A \Rightarrow B$
	\end{itemize}
	\item Indirekter Beweis:
	\begin{itemize}
		\item Annahme das Aussage korrekt, durch Negation der Annahme muss der Schluss falsch sein
	\end{itemize}
	\item Vollständige Induktion
	\begin{itemize}
		\item Beweise für Aussagen über die natürlichen Zahlen $\mathbb{N}$
		\item Basierend auf den Peano-Axiomen für $\mathbb{N}$
	\end{itemize}
\end{itemize}
\end{frame}

\begin{frame}{Beweisregeln}
\begin{itemize}
	\item Abtrennungsregel
	\item Fallunterscheidung
	\item Kettenschluss
	\item Indirekter Beweis
	\begin{itemize}
		\item Sind $A \Rightarrow B$ und $A \Rightarrow \neg B$ allgemeingültig, so ist $\neg A$ allgemeingültig
		\item Korrektheit folgt aus der Allgemeingültigkeit von $((A \Rightarrow B) \land (A \Rightarrow (\neg B))) \Rightarrow (\neg A)$
	\end{itemize}
	\item Kontraposition:  Ist $A \Rightarrow B$ allgemeingültig, so ist $(\neg B) \Rightarrow (\neg A)$ allgemeingültig
	\begin{itemize}
		\item Korrektheit folgt aus der Allgemeingültigkeit von $(A \Rightarrow B) \Rightarrow ((\neg B) \Rightarrow (\neg A))$.
	\end{itemize}
\end{itemize}
\end{frame}

\subsection{Vollständige Induktion}
\begin{frame}{Vollständige Induktion}
\begin{itemize}
	\item Drei Teile: 
	\begin{itemize}
		\item Induktionsanfang (IA) \& Induktionsannahme
		\item Induktionsschritt (IS)
		\item Induktionsschluss
	\end{itemize}
\end{itemize}
\end{frame}

\begin{frame}{Beispiel: Vollständige Induktion}
\begin{theorem}
$$\forall n (n \in \mathbb{N}_0 \rightarrow 2^0 + 2^1 + \dots 2^n = 2^{n+1}-1)$$
\end{theorem}
\end{frame}

\begin{frame}
\begin{proof}
Prädikat: $\varphi(n) \equiv (2^0 + 2^1 + \dots 2^n = 2^{n+1}-1)$
\begin{enumerate}
	\item Induktionsanfang (IA): $\varphi(0)$ soll gelten
$2^0 = 2^{0+1}-1 \Leftrightarrow 1 = 1 \surd$\\
	\item Induktionsschritt (IS):
\begin{align*}
\varphi(n) & \Rightarrow \varphi(n^+)\\
2^0 + 2^1 + \dots 2^n + 2^{n+1} &= 2^{(n+1)+1}-1\\
\textbf{Anm.: } 2^0 + 2^1 + \dots 2^n &= 2^{n+1} -1  \\
\Leftrightarrow  2^{n+1} -1 + 2^{n+1} &= 2^{(n+1)+1}-1\\
\textbf{Anm.: } a^n + a^m &= 2^{n+m}\\
\Leftrightarrow  2^{n+2} -1 &= 2^{(n+2)}-1 \surd
\end{align*}
\end{enumerate}
\end{proof}
\end{frame}

\begin{frame}
\begin{proof}
Prädikat: $\varphi(n) \equiv (2^0 + 2^1 + \dots 2^n = 2^{n+1}-1)$
\begin{enumerate}
	\item Induktionsanfang: $\varphi(0)$ soll gelten
$2^0 = 2^{0+1}-1 \Leftrightarrow 1 = 1 \surd$\\
	\item Induktionsschritt:
\begin{align*}
\varphi(n) & \Rightarrow \varphi(n^+)\\
2^0 + 2^1 + \dots 2^n + 2^{n+1} &= 2^{(n+1)+1}-1\\
\Leftrightarrow  2^{n+1} -1 + 2^{n+1} &= 2^{(n+1)+1}-1\\
\Leftrightarrow  2^{n+2} -1 &= 2^{(n+2)}-1 \surd
\end{align*}
	\item Induktionsschluss:
$$\text{nach IA und IS } \Rightarrow \varphi(n) (\forall n(\varphi(n)))$$
\end{enumerate}
\end{proof}
\end{frame}

\subsection{Strukturelle Induktion}
\begin{frame}{Strukturelle Induktion}
\begin{itemize}
	\item Vollständige Induktion ist eine Spezialfall der strukturellen Induktion
\end{itemize}
\end{frame}






\section*{Quellen}
\appendix
\begin{frame}[allowframebreaks]
  \frametitle<presentation>{Quellen}
\printbibliography
\end{frame}
\end{document}