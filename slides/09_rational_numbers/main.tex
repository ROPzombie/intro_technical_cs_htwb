\documentclass[12pt%
%,draft%
,aspectratio=169%
]{beamer}
%
\usepackage{fontspec}
\defaultfontfeatures{Ligatures=TeX}
%\setsansfont{Liberation Sans}
\usepackage{polyglossia}
\setdefaultlanguage{ngerman}
% Alternative template for talks of the Freie Universität Berlin.
% Created by Leonard R. König, <leonard.koenig@fu-berlin.de> following the
% guidelines on www.fu-berlin.de/cd
%
% (c) Leonard König, CC BY 4.0
%
% This template was written against UTF-8 capable LaTeX engines, specifically
% LuaLaTeX.

% Trying to get rather close to the ppt/odp template:
%  http://www.fu-berlin.de/sites/cd/downloads_container/PowerPoint_Praesentation_Anleitung.pdf

%%% font styles
\setbeamerfont{frametitle}{series=\bfseries}
\setbeamerfont{footline}{series=\bfseries}
\setbeamerfont{headline}{series=\bfseries}
\setbeamerfont{alerted text}{series=\bfseries}
%%%

% colordefs
\definecolor{fu_darkblue}{RGB}{0,51,102}
\definecolor{fu_seablue}{RGB}{0,102,204}
\definecolor{fu_lightblue}{RGB}{204,214,224}
\definecolor{fu_green}{RGB}{153,204,0}
\definecolor{fu_lightgrey}{RGB}{128,128,128}
\definecolor{fu_grey}{RGB}{95,95,95}
%
\definecolor{fu_red}{RGB}{204, 0, 0} % red text (used by \alert)
%%% end colordefs

%%% colors
\setbeamercolor*{title}{fg=fu_darkblue}
\setbeamercolor*{subtitle}{fg=fu_seablue}
\setbeamercolor*{frametitle}{fg=fu_darkblue}
\setbeamercolor*{footline}{fg=fu_grey,bg=fu_lightblue}
\setbeamercolor*{headline}{fg=fu_grey}

\setbeamercolor*{normal text}{fg=black}
\setbeamercolor*{alerted text}{fg=fu_red}
\setbeamercolor*{example text}{fg=fu_green}
\setbeamercolor*{structure}{fg=fu_darkblue}

\setbeamercolor*{block title}{fg=white,bg=black!50}
\setbeamercolor*{block title alerted}{fg=white,bg=black!50}
\setbeamercolor*{block title example}{fg=white,bg=black!50}

\setbeamercolor*{block body}{bg=black!10}
\setbeamercolor*{block body alerted}{bg=black!10}
\setbeamercolor*{block body example}{bg=black!10}

\setbeamercolor{bibliography entry author}{fg=fu_darkblue}

\setbeamercolor{item}{fg=fu_darkblue}
\setbeamercolor{navigation symbols}{fg=fu_lightgrey,bg=fu_grey}
%%% end colors

%%% title page
% Display logo (if exists) and right next to it, put our title + subtitle
\defbeamertemplate*{title page}{fu_titlepage}
{%
	\hskip .3\textheight
	\begin{minipage}[.4\textheight]{\textwidth}
		\begin{minipage}[.4\textheight]{0.25\textwidth}
			\inserttitlegraphic
		\end{minipage}%
		\begin{minipage}[.4\textheight]{0.75\textwidth}
			\begin{beamercolorbox}{title}
				\usebeamerfont{title}\inserttitle\par%
			\end{beamercolorbox}
			\vfill
			\ifx\insertsubtitle
				\@empty%
			\else
				\begin{beamercolorbox}{subtitle}
					\usebeamerfont{subtitle}\insertsubtitle\par
				\end{beamercolorbox}
			\fi
		\end{minipage}
	\end{minipage}%
	\hskip .3\textheight
}
%%% end title page

%%% headline
% display title, author and institute on the left;
% logo on the right.
\newcommand{\headlinetext}
{%
	\inserttitle\\[0.3em]%
	\insertauthor, %
	\insertshortinstitute
}
\newlength{\headlinewidth}
\setlength{\headlinewidth}{\paperwidth}
\addtolength{\headlinewidth}{-2\marginparsep}
\setbeamertemplate{headline}
{%
	\begin{beamercolorbox}[wd=\paperwidth]{headline}%
		\vskip5pt
		{\hspace*{\marginparsep}}%
		\parbox{.5\headlinewidth}
		{%
			\usebeamertemplate{title in head/foot}%
			\headlinetext%
		}%
		\begin{minipage}{.5\headlinewidth}%
			\hfill\usebeamertemplate*{logo}
		\end{minipage}%
		{\hspace*{\marginparsep}}%
	\end{beamercolorbox}%
}
%%% end headline

%%% footline
% title + date on the left, frame number on the right
\newcommand{\footlinetext}
{%
	\usebeamerfont{shorttitle}\insertshorttitle, %
	\usebeamerfont{shortdate}\insertshortdate
}
\setbeamertemplate{footline}
{%
	\begin{beamercolorbox}{footline}
		\vskip2pt
		\hspace{\marginparsep}%
		\footlinetext\hfill%
		\insertframenumber%
		\hspace{\marginparsep}
		\vskip2pt
	\end{beamercolorbox}%
}
%%% end footline

% don't use default templates for sidebars
\setbeamertemplate{sidebar right}{}
\setbeamertemplate{sidebar left}{}
\setbeamertemplate{title page}[fu_titlepage]
\usepackage{amsmath}
\usepackage{amsfonts}
\usepackage{amssymb}
\usepackage{graphicx}
\usepackage{algorithm}
\usepackage[noend]{algpseudocode}
%\usepackage{algorithmic}
\usepackage{tikz}
\usetikzlibrary{arrows,shapes,automata,petri,positioning,calc}
\usepackage{graphicx}
\usepackage{subfig}
\usepackage{pgfplots}
\usepackage{venndiagram}
\usepackage{ stmaryrd }
\usepackage[normalem]{ulem}
\usepackage{circuitikz}
\usepackage{bohr}
\usepackage{csquotes}

\setbeamercolor{block title}{use=structure,fg=white,bg=structure.fg!75!black}
\setbeamercolor{block body}{parent=normal text,use=block title,bg=block title.bg!10!bg}


\author{Benjamin Tröster}
\title[Zahlendarstellung]{Zahlendarstellung}
%\subtitle[Markov Models]{...}
%\pgfdeclareimage{titlegraphic}{../res/dwarf_logo2.png}
%\titlegraphic{\pgfuseimage{titlegraphic}}
%\date{}
%\subject{}
%
% FU settings
\institute[HTW Berlin]{Hochschule für Technik und Wirtschaft Berlin}
%\pgfdeclareimage[height=0.9cm]{logo}{../res/dwarf_logo}
%\logo{\pgfuseimage{logo}}
%
\usepackage[
backend=biber,
citestyle=alphabetic,bibstyle=authoryear
]{biblatex}
\addbibresource{sources.bib}


\begin{document}

\begin{frame}
\titlepage
\end{frame}

\begin{frame}{Fahrplan}
\tableofcontents[hideothersubsections]
\end{frame}

\section{Einleitung}
\begin{frame}{Heute}
\begin{itemize}
	\item Coronabedingt: Sprung von Schaltkreisen und Transistoren zur Zahlendarstellung
	\item Ziel: Wir bauen ein Rechenwerk (ALU) aus Schaltkreisen mithilfe von Gattern
	\item Zwischenziel: Wie können wir die Zahlen im Rechner darstellen?
	\item Darstellung der natürlichen Zahlen $\mathbb{N} \quad \surd$
	\item Darstellung der ganzer Zahlen $\mathbb{Z} \quad \surd$
\end{itemize}
\end{frame}



\section{Rationale Zahlen}
\begin{frame}{Die rationalen Zahlen $\mathbb{Q}$ (anschaulich)}
$$
	\mathbb{Q} = \Bigg\{ \frac{a}{b} | a, b \in \mathbb{Z}, b \neq 0 \Bigg\} 
$$
Bruchrechenregeln
$$
	\frac{a}{b} + \frac{a'}{b'} = \frac{ab' + a'b}{bb'} \qquad \qquad \frac{a}{b} \cdot \frac{a'}{b'} = \frac{aa'}{bb'}
$$
Konsequenz
$$
	\frac{a}{b} = \frac{a'}{b'}  \qquad \Leftrightarrow \qquad ab' = a'b
$$

\end{frame}

\begin{frame}{Die rationalen Zahlen $\mathbb{Q}$ (konstruktiv)}
$$
	\mathbb{Q} = \Bigg\{ \frac{a}{b} | a, b \in \mathbb{Z}, b \neq 0 \Bigg\} 
$$
\textcolor{blue}{Problem}
\begin{itemize}
	\item Im Allgemeinen hat $x \cdot b = a$ keine Lösung $x \in \mathbb{Z}$.
\end{itemize}
\textcolor{blue}{Konstruktion von $\mathbb{Q}$}
	\begin{itemize}
		\item Abschluss von $\mathbb{Z}$ unter Division
		\item Äquivalenzklassen von Paaren $(a, b)$ mit $a, b \in \mathbb{Z}, b \neq 0$
	\end{itemize}

\end{frame}

\begin{frame}{Darstellung von $\mathbb{Q}$}
\begin{theorem}
Jede Zifferndarstellung von $\mathbb{Q}$ induziert eine Zifferndarstellung von $\mathbb{Q}$.\\
Ziffernmenge: $\mathcal{Z} \cup \{-\} \cup \{/\}$
\end{theorem}
\textcolor{blue}{Folgerung}
\begin{itemize}
	\item $\mathbb{Q}$ ist abzählbar
\end{itemize}
\textcolor{blue}{Beispiele}
\begin{itemize}
	\item Dezimalsystem
	\item Dualsystem
\end{itemize}
\end{frame}

\subsection{q-adische Brüche}
\begin{frame}{$q$-adische Brüche}
$$
	z_n \ldots z_0, z_{-1} \ldots z_{-m} = \sum_{-m}^n z_i q^i, \qquad z_i \in \{0, \ldots, q-1\}, n,m \in \mathbb{N}
$$
Beispiele:
\begin{itemize}
	\item Dezimalbrüche: $q = 10$
	\item Dualbruch: $q = 2$
\end{itemize}
\begin{theorem}
Jeder Dualbruch ist ein Dezimalbruch, nicht umgekehrt.
\end{theorem}
\begin{theorem}
Jeder $q$-adische Bruch ist eine rationale Zahl, nicht umgekehrt.
\end{theorem}
\end{frame}

\subsection*{Periodische Dezimalbrüche}
\begin{frame}{Periodische Dezimalbrüche}
Beispiel:\\
Periodischer Dezimalbruch (Periodenlänge 3): $0,1234234 \ldots = 0,\overline{1234}$

\textcolor{blue}{Geometrische Reihe}
$$
	\sum_{i=0}^\infty q^{-i} = \lim_{m \to \infty} \sum_{i=0}^m q^{-i} = \lim_{m \to \infty} \frac{1 -q^{-(m+1)}}{1 - q^{-1}} = \frac{1}{1 - q^{-1}}
$$
\begin{theorem}
Jeder periodische Dezimalbruch ist eine rationale Zahl und umgekehrt.
\end{theorem}
\textcolor{blue}{Eindeutigkeit}
\begin{itemize}
	\item Im Allgemeinen ist die Darstellung nicht eindeutig: $1,0 = 0,9$
	\item Eindeutigkeit erzwingen: $0$ ist verboten.
\end{itemize}
\end{frame}

\begin{frame}{Praktische Realisierung im Rechner}
\begin{itemize}
	\item Darstellung als Paar von integer-Zahlen
	\begin{itemize}
		\item integer = Ganzzahldarstellung im Rechner
		\item Länge muss variable sein
		\item Aufwand für Rechenoperationen nicht a priori bekannt (Kürzen!)
	\end{itemize}
	\item Keine standardisierte Hardware-Unterstützung
	\begin{itemize}
		\item Spezialanwendungen (Schnitterkennung in der Computergraphik)
		\item Symbolik-Programme (Maple, Mathematica, ...)
		\item Aufwendig und langsam
	\end{itemize}
\end{itemize}
\end{frame}


\section*{Quellen}
\appendix
\begin{frame}[allowframebreaks]
  \frametitle<presentation>{Quellen}
\printbibliography
\end{frame}
\end{document}