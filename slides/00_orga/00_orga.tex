\documentclass[xcolor=dvipsnames,aspectratio=169]{beamer}

\usepackage{fontspec}
\defaultfontfeatures{Ligatures=TeX}
%\setsansfont{Liberation Sans}
\usepackage{polyglossia}
\setdefaultlanguage{german}

\usetheme{Berlin}
\usecolortheme[named=LimeGreen]{structure}
\usepackage{beamerthemesplit} % kam neu dazu
	
\usepackage{amsmath}
\usepackage{graphicx}
\graphicspath{{pictures/}}
\usepackage{amssymb}
\usepackage{amsfonts}
\usepackage{caption}
\usepackage{multimedia}
\usepackage{tikz}
\usepackage{listings}
\usepackage{acronym}
\usepackage{uhrzeit}
\usepackage{lmodern}
\usepackage{multicol}
\usepackage{csquotes}


\definecolor{pblue}{rgb}{0.13,0.13,1}
\definecolor{pgreen}{rgb}{0,0.5,0}
\definecolor{pred}{rgb}{0.9,0,0}
\definecolor{pgrey}{rgb}{0.46,0.45,0.48}

\newcounter{countitems}
\newcounter{nextitemizecount}
\newcommand{\setupcountitems}{%
  \stepcounter{nextitemizecount}%
  \setcounter{countitems}{0}%
  \preto\item{\stepcounter{countitems}}%
}
\makeatletter
\newcommand{\computecountitems}{%
  \edef\@currentlabel{\number\c@countitems}%
  \label{countitems@\number\numexpr\value{nextitemizecount}-1\relax}%
}
\newcommand{\nextitemizecount}{%
  \getrefnumber{countitems@\number\c@nextitemizecount}%
}
\newcommand{\previtemizecount}{%
  \getrefnumber{countitems@\number\numexpr\value{nextitemizecount}-1\relax}%
}

\makeatother    
\newenvironment{AutoMultiColItemize}{%
\ifnumcomp{\nextitemizecount}{>}{3}{\begin{multicols}{2}}{}%
\setupcountitems\begin{itemize}}%
{\end{itemize}%
\unskip\computecountitems\ifnumcomp{\previtemizecount}{>}{3}{\end{multicols}}{}}

\lstset{
    escapeinside={(*}{*)}
}

\lstdefinestyle{Java}{
  showspaces=false,
  showtabs=false,
  tabsize=2,
  breaklines=true,
  showstringspaces=false,
  breakatwhitespace=true,
  commentstyle=\color{pgreen},
  keywordstyle=\color{pblue},
  stringstyle=\color{pred},
  basicstyle=\footnotesize\ttfamily,
  numbers=left,
  numberstyle=\tiny\color{gray}\ttfamily,
  numbersep=7pt,
  %moredelim=[il][\textcolor{pgrey}]{$$},
  moredelim=[is][\textcolor{pgrey}]{\%\%}{\%\%},
  captionpos=b
}

\lstdefinestyle{basic}{  
  basicstyle=\footnotesize\ttfamily,
  breaklines=true
  numbers=left,
  numberstyle=\tiny\color{gray}\ttfamily,
  numbersep=7pt,
  backgroundcolor=\color{white},
  showspaces=false,
  showstringspaces=false,
  showtabs=false,
  frame=single,
  rulecolor=\color{black},
  captionpos=b,
  keywordstyle=\color{blue}\bf,
  commentstyle=\color{gray},
  stringstyle=\color{green},
  keywordstyle={[2]\color{red}\bf},
}


\lstdefinelanguage{custom}
{
morekeywords={public, void},
sensitive=false,
morecomment=[l]{//},
morecomment=[s]{/*}{*/},
morestring=[b]",
}


\lstdefinestyle{BashInputStyle}{
  language=bash,
  showstringspaces=false,
  basicstyle=\small\sffamily,
  numbers=left,
  numberstyle=\tiny,
  numbersep=5pt,
  frame=trlb,
  columns=fullflexible,
  backgroundcolor=\color{gray!20},
  linewidth=0.9\linewidth,
  xleftmargin=0.1\linewidth
}

%Logo in the upper right just change if you know what you are doing^^
\addtobeamertemplate{frametitle}{}{%
\begin{tikzpicture}[remember picture,overlay]
\node[anchor=north east,yshift=2pt] at (current page.north east) {\includegraphics[height=1.8cm]{htw}};
\end{tikzpicture}}

\begin{document}
\bibliographystyle{alpha}
\title{Technische und logische Grundlagen der Informatik\\Wintersemester 2021/22}
\subtitle{Organisatorisches\\

\href{mailto:Benjamin.Troester@HTW-Berlin.de}{Benjamin.Troester@HTW-Berlin.de}\\
		PGP: ADE1 3997 3D5D B25D 3F8F 0A51 A03A 3A24 978D D673 }

\author{Benjamin Tröster}

\date{}

\begin{frame}
\titlepage
\end{frame}

\section*{Road-Map}
\begin{frame}
\frametitle{Road-Map}
\begin{multicols}{2}
  \tableofcontents
\end{multicols}
\end{frame}

\section{Disclamer}
\begin{frame}
	\frametitle{Disclamer I}
	\begin{itemize}
		\item Objektive Kritik ist immer willkommen!
		\item Wenn sie Fehler oder Anmerkungen haben, tragen sie diese ruhig vor.
		\item Sie wollen ehrliches Feedback genauso wie ich
		\item Wenn Sie Anregungen, Verbesserungsvorschläge haben, nehme ich diese ebenfalls gerne an.
	\end{itemize}
\end{frame}

\section{Organisatorisches}
\subsection{Grundsätzliches}
\begin{frame}
	\frametitle{Grundsätzliches}
	\begin{itemize}
		\item Vorlesung: ab Oktober 2021 06.10.2021
		\item (Mglw.: Meeting/Übung via Moodle: Telco-Plugin \emph{BigBlueButton (BBB)}
		\item Vorlesungs \& Übungszeit $\Rightarrow$ LSF
		\begin{itemize}
			\item Beachten sie Gruppentermine für die Übungen im LSF
		\end{itemize}
		\item Skript, Übungsblätter, Folien, Literatur \& Links etc. $\Rightarrow$ \url{moodle.htw-berlin.de}
		\item Übungen: Zweiwöchentlicher Tonus 
		\item Erste Übung: 13.10.2021
	\end{itemize}
\end{frame}

\begin{frame}
	\frametitle{Klausur}
		\begin{itemize}
			\item Dieses Semester: Prüfung Wahrscheinlich im Klausurformat	
			\begin{itemize}
				\item Schriftlich
				\item Dauer: 90 Minuten
			\end{itemize}
			\item Klausurinhalt: 50\% Übung, 50\% Vorlesung
		\end{itemize}
\end{frame}

\begin{frame}
	\frametitle{Vorlesungsablauf}
	\begin{itemize}
		\item Beginn: Re­ka­pi­tu­la­ti­on \& Wiederholung wichtigster Inhalte letzten Vorlesung
		\begin{itemize}
			\item Stellen sie Frage!
		\end{itemize}
		\item Vorlesung Mischung aus Slides bzw. Tafel
		\begin{itemize}
			\item Skript: Allgemeine Zusammenfassung wesentlicher Inhalte
			\item Folien: Visualisierung
			\item Tafel: Praktische Beispiele, Erläuterungen
		\end{itemize}
	\end{itemize}
\end{frame}

\begin{frame}
	\frametitle{Übungsablauf}
	\begin{itemize}
		\item Beginn mit Nachbesprechung letzter Übung:
		\begin{itemize}
			\item Grundsätzliche ihre \enquote{Show}, sie bestimmen hier den wesentlichen Lauf
			\item Fragen aus den Hausaufgaben
			\item Besprechung aktueller Übungszettel
			\item Lösen von Aufgaben
		\end{itemize}
	\end{itemize}
\end{frame}

\subsection{ECTS}
\begin{frame}
	\frametitle{Arbeitsaufwand -- ECTS}
	\begin{itemize}
		\item Modul Netzwerke: 5 ECTS (European Credit Transfer System) -- manchmal auch LP/CP (Leistungspunkte)
			\begin{itemize}
				\item 1 ECTS $\widehat{=}$ 30h
				\item Workload TlGI: 150h/Semester oder 37,5h/Monat oder $\sim$ 9,375h/Woche
				\item 2 SWS \footnote{Semesterwochenstunden} VL + 1 SWS Übung $\rightarrow$ 3 SWS oder 135 min
				\item D.h. restlichen $\sim$\textbf{6h/Woche}
				\begin{itemize}
					\item Vorbereitung \& Nachbereitung der Vorlesung
					\item Hausaufgaben, Recherche, ...
				\end{itemize}
			\end{itemize}
			\item Total Workload pro Semester: 30 CP $\widehat{=}$ 900h $\rightarrow$  $\sim$56.25/Woche
	\end{itemize}
\end{frame}

\section{Motivation}
\subsection{Studien- \& Prüfungsordnung AMB30/2012}
\begin{frame}
	\frametitle{B12 Technische und logische Grundlagen der Informatik}
	Studien- \& Prüfungsordnung AMBl. 12/2021 S. 207\\
	\begin{itemize}
		\item Boolesche Algebra, Logikgatter, kombinatorische und sequentielle Schaltungen \& Schaltwerke
		\item Zeichenkodierung, Zahlendarstellung und Rechenarithmetik,
		\item Signale im Zeit und Frequenzbereich
	\end{itemize}
\end{frame}

\begin{frame}
	\frametitle{Nachdem sie das Modul erfolgreich absolviert haben, können die Studierenden:}
	\vspace{-0.7cm}
	\begin{itemize}
		\item logischen Ausdrücke spezifizieren und vereinfachen, 
		\item Bitweise Operatoren anwenden
		\item Synchrone Automaten beschreiben und simulieren
		\item Mit relevanten Verfahren zur Signalverarbeitung umgehen
		\item Im Rahmen des Fachgebietes bekannte Vorgehen adaptieren und an neue Anforderungen anpasse
		\item fachliche Fragestellungen verstehen und Lösungen vorschlagen. 
	\end{itemize}
\end{frame}

\end{document}